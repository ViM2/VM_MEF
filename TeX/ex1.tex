\chapter{Exemple de formulation variationnelle d'un problème de Neumann}
\begin{abstract}
Dans cet exemple, portant sur formulation d'un problème de Neumann,\index[aut]{Neumann (Carl Gottfried), 1832-1925, Allemand}
nous allons essayer de montrer comment la réflexion mathématique se
fait et évolue «au fil de l'eau» pour transformer un problème initial donné et obtenir
les bonnes conditions d'existence et d'unicité de la solution sur les «espaces qui vont
bien» (et qui eux, feront ensuite l'objet d'une discrétisation numérique).
\end{abstract}

Supposons que nos «investigations» sur un problème nous aient
conduit à sa formulation sous forme d'un problème de Neumann\index[aut]{Neumann (Carl Gottfried), 1832-1925, Allemand}
(avec~$f\in L^2(\Omega)$, ce qui n'est pas une condition si forte finalement, et c'est ce que l'on aimerait avoir):
\begin{equation}\left\{
\begin{aligned}
-&\Delta u = f && \text{ dans } \Omega\\
&\frac{\partial u}{\partial n}=0 && \text{ sur }\Gamma=\partial\Omega
\end{aligned}
\right.
\end{equation}
Nous allons donc nous poser la question de l'existence et de l'unicité de la solution de ce problème.

\medskip
\section{Étude directe de l'existence et l'unicité de la solution}
\textcolorgreen{Sans être magicien, il doit sembler à peu près évident qu'une condition nécessaire d'existence
doit être que~$f\in L^2_0(\Omega)$, et que les conditions imposées semblent un peu courtes
pour assurer l'unicité.}

On voit en effet que l'existence implique que
\begin{equation}\dint_\Omega -\Delta u = \dint_\Gamma \frac{\partial u}{\partial n}=0\end{equation}
qui conduit à:\begin{equation}\dint_\Omega f =0\end{equation} i.e.~$f\in L^2_0(\Omega)$.
Donc \textcolorblue{$f\in L^2_0(\Omega)$ est bien une condition nécessaire d'existence.}

De même, d'après ce qui précède, on voit que si~$u$ est solution, alors
$u+c$ est solution, ce qui prouve qu'\textcolorblue{il n'y a pas unicité de la solution.}

\medskip
\section{Formulation variationnelle}
\textcolorgreen{Nous allons écrire la formulation variationnelle de notre problème dans~$H^1(\Omega)$.
Nous regarderons si nous retrouvons la même condition d'existence.
Nous nous demanderons ensuite si, en nous restreignant à un certain espace, il n'est
pas possible d'obtenir l'unicité de la solution.}

\medskip
Nous commençons donc par écrire notre problème sous la forme:
\begin{equation}
\dint_\Omega -\Delta u\cdot v = \dint_\Omega \nabla u\cdot \nabla v - \dint_\Gamma \frac{\partial u}{\partial n} v
\end{equation}
et comme le dernier terme est nul (condition aux limites), il reste:
\begin{equation}
\dint_\Omega -\Delta u\cdot v = \dint_\Omega \nabla u\cdot \nabla v
\end{equation}
Comme nous voulons la formulation variationnelle dans~$H^1(\Omega)$ (i.e.~$u\in H^1(\Omega))$, il vient:
\begin{equation}
\dint_\Omega \nabla u\cdot \nabla v = \dint_\Omega f v, \quad \forall v\in H^1(\Omega)
\end{equation}
\medskip
Si nous considérons~$u=u_0+c$ et~$v=v_0+d$, alors il vient:
\begin{equation}
\dint_\Omega \nabla u_0\cdot\nabla v_0 = \dint_\Omega f v_0 + d\dint_\Omega f, \quad \forall v_0\in H^1(\Omega),\quad \forall d
\end{equation}
ce qui implique, pour que ce soit vrai pour tout~$d$ que~$\int_\Omega f=0$, et on retrouve
bien que \textcolorblue{$f\in L^2_0(\Omega)$ est une condition nécessaire d'existence.}

\medskip
Mézalor, il est naturel de considérer~$u_0\in H^1(\Omega)\cap L^2_0(\Omega)$ \textcolorgreen{(car
on veut que~$u$ et~$v$ vivent dans le même espace pour pouvoir appliquer nos «méthodes»)}, et
la formulation précédente donne:
\begin{equation}
\dint_\Omega \nabla u_0\cdot\nabla v_0 = \dint_\Omega f v_0,
\quad \forall v_0\in H^1(\Omega)\cap L^2_0(\Omega)
\end{equation}
i.e. nous sommes ammenés à considérer l'espace \textcolorblue{$V=H^1(\Omega)\cap L^2_0(\Omega)$.}

\medskip
L'espace~$V$ est un espace de Hilbert\index[aut]{Hilbert (David), 1862-1943, Allemand}
normé par~$\|\cdot\|_{H^1(\Omega)}$.
\begin{itemize}
\item~$v\in V \Rightarrow v\in H^1(\Omega)$ tel que~$\int_\Omega v=0$ et l'application~$v\in H^1(\Omega) \mapsto \int_\Omega v$ est continue sur~$H^1(\Omega)$, donc~$V$ est un sous-espace fermé de~$H^1$, donc est un Hilbert. \textcolorgreen{Et donc
	maintenant que l'on sait que~$V$ est un Hilbert, on aimerait bien appliquer Lax-Milgram\index[aut]{Lax (Peter), 1926-, Américain}\index[aut]{Milgram (Arthur Norton), 1912-1961, Américain}\index{théorème!de Lax-Milgram}
	(voir~\ref{Sec:LaxMil}),
	c'est pourquoi il faut s'intéresser à la forme bilinéaire définissant le problème. C'est ce
	qui suit.}
  \item Si l'on considère la \textcolorblue{forme bilinéaire}: \begin{equation} a(u_0, v_0) = \int_\Omega \nabla u_0\cdot\nabla v_0\end{equation}
	alors on voit que \begin{equation}a(u_0, v_0) \le |u_0|_1 \overline{|v_0|}_1 \le \| u_0\|_{H^1} \|v_0\|_{H^1}\end{equation}
	et donc~$a$ est \textbf{continue}.
  \item~$(v_0, v_0) =|v_0|^2_1$. Or on a l'inégalité de type Poincaré suivante:~$\exists c(\Omega)>0$ tel que: 
	\begin{equation}|\varphi|_{0,\Omega} \le c(\Omega)|\varphi|_{1,\Omega}, \quad \forall \varphi\in V\end{equation}
	Cela conduit à: \begin{equation}\dfrac{|v_0|^2_1}{|v_0|^2_1+|v_0|^2_0} \ge \dfrac1{1+c^2(\Omega)},\quad 
	\forall v\in V\end{equation} Donc~$a$ est \textbf{V-elliptique}.
\end{itemize}
\textcolorblue{Le théorème de Lax-Milgram\index{théorème!de Lax-Milgram}\index[aut]{Lax (Peter), 1926-, Américain}\index[aut]{Milgram (Arthur Norton), 1912-1961, Américain}
permet donc de conclure à l'existence et à l'unicité
de la solution~$u_0\in V$}.

\medskip
\section{Formulation mixte duale}
\textcolorgreen{On va regarder maintenant ce qui se passe si l'on exprime le problème sous forme mixte, i.e. en
se servant de deux variables. C'est une voie fréquente pour «simplifier» les conditions
aux limites.}

En effet, plutôt que considérer~$\frac{\partial u}{\partial n}=0$ sur~$\Gamma=\partial\Omega$,
on peut interpréter cela comme une pression nulle sur les bords, i.e.~$p.n=0$ sur~$\Gamma$.

\medskip
Le problème considéré se formule alors de la manière suivante:
\begin{equation}\left\{
\begin{aligned}
-&\Delta u +p =0 && \text{ loi de comportement dans } \Omega\\
&\dive p = -f && \text{ loi de conservation dans } \Omega\\
& p\cdot n=0 && \text{ condition aux limites sur }\Gamma
\end{aligned}
\right.
\end{equation}

On peut alors écrire la loi de conservation:
\begin{equation}
\dint_\Omega (\dive p) v = -\dint_\Omega fv,\quad \forall v\in L^2(\Omega) \text{ (à priori)}
\end{equation}
et la loi de comportement:
\begin{equation}
0=\dint_\Omega (p-\nabla u)\cdot q = \dint_\Omega p\cdot q + \dint_\Omega u \dive q=\langle u,q\cdot n\rangle
\end{equation}
On n'a pas d'information sur~$u$.
Par contre, on a un terme~$p\cdot n=0$.
Et il serait bien de disposer de son «symétrique», i.e. d'un terme~$q.n=0$ qui serait imposé
par le choix de l'espace dans lequel vivrait~$u$.
\textcolorgreen{Ce que l'on a en tête c'est bien sûr de pouvoir utiliser le théorème
de Brezzi.}\index[aut]{Brezzi (Franco), 1945-, Italien}\index{théorème!de Brezzi}

\medskip
En correspondance avec les notations du paragraphe~\ref{Sec:Brezzi}, on choisit alors les espaces:
\textcolorblue{$H=\{ q\in H(\dive;\Omega); q\cdot n_{|\Gamma}=0\}$}
et~$M=L^2(\Omega)$.
Le problème est donné, dans la formulation de Brezzi,\index{théorème!de Brezzi}\index[aut]{Brezzi (Franco), 1945-, Italien}
par les deux formes bilinéaires
\begin{equation}a(p,q)=\dint_\Omega p\cdot q\quad\text{ et }\quad b(v,q)=\dint_\Omega v\dive q\end{equation}
Quant aux formes linéaires définissant le problème, la première est nulle et la
seconde vaut:\begin{equation}-\dint_\Omega fv\end{equation}

\medskip
Maintenant que nous nous sommes rapprochés du cadre du théorème de Brezzi,\index{théorème!de Brezzi}\index[aut]{Brezzi (Franco), 1945-, Italien}
il va nous falloir vérifier si toutes les conditions sont vérifiées.
\textcolorgreen{Toutefois, d'après ce qui a été vu dans les paragraphes précédents,
il semble naturel de se demander s'il faudra bien conserver~$L^2(\Omega)$, ou s'il faudra passer
à~$L^2_0(\Omega)$, et ce qu'il se passe quand~$u=u_0+c$.}

\medskip
Pour l'instant, on a:
\begin{equation}\dint_\Omega v \dive p = -\dint_\Omega fv,\quad\forall v\in L^2(\Omega)\end{equation}
Si l'on prend~$v=1$, alors il vient:
\begin{equation}\dint_\Omega \dive p = -\dint_\Gamma p\cdot n =0
\end{equation}
ce qui
implique que~$\int_\Omega f=0$, soit~$f\in L^2_0(\Omega)$
\medskip
On a également:
\begin{equation}\int_\Omega p\cdot q + \dint_\Omega u \dive q =0,\quad\forall q\in H\end{equation}
Or~$q\in H$ implique~$\int_\Omega \dive q =0$, ce qui implique aussi que si~$u$ est solution, alors
$u+c$ l'est aussi.

Encore une fois il n'y a pas uncitié de la solution. Pour l'imposer, il faut donc une condition sur~$u$, comme
par exemple~$\int_\Omega u=0$, i.e.~$u\in L^2_0(\Omega)$.
Si de plus~$\dive p\in L^2_0(\Omega)$ (et avec~$f\in L^2_0(\Omega)$ comme vu au dessus), alors ce
implique que:
\begin{equation}\dint_\Omega (\dive p + f)(v_{L^2_0}+c) = \dint_\Omega (\dive p+f) v_{L^2_0}\end{equation}
\medskip
Donc le choix \textcolorblue{$M=L^2_0(\Omega)$} semble un choix convenable à priori.

\medskip
Il reste alors maintenant à \textbf{vérifier que toutes les conditions du théorème de Brezzi}\index[aut]{Brezzi (Franco), 1945-, Italien}\index{théorème!de Brezzi}
sont bien satisfaites pour pouvoir conclure à l'existence et à l'unicité de la solution.
Ces conditions sont:
\begin{itemize}
  \item~$H$ et~$M$ sont des Hilbert;
  \item~$a$, $b$ et~$f$ sont continues;
  \item~$a$ est V-elliptique;
  \item condition inf-sup sur~$b$.
\end{itemize}

\begin{description}
\item[$H$ est un Hilbert]
Pour montrer que~$H$ est un Hilbert, on s'appuie sur le fait que l'application:
$\varphi: q\in H(\dive;\Omega)\mapsto q\cdot n\in H^{-1/2}(\Gamma)$ est continue
(car~$\|q,0\|_{-1/2,\Gamma}\le \|q\|_{H(\dive;\Omega)}$).

Il en résulte alors que~$H=\varphi^{-1}(0)$ est fermé et que~$H(\dive;\Omega)$ étant un
Hilbert, c'est également un Hilbert pour la norme de~$H(\dive)$.

\medskip
\item[$M$ est un Hilbert]
Pour montrer que~$M$ est un Hilbert, on s'appuie sur le fait que l'application:
$\varphi: v\in L^2(\Omega)\mapsto \int_\Omega v=0$ est continue
car~$|\int v| \le \sqrt{|\Omega|} |v_{L^2(\Omega)}|$.

Il en résulte que~$M=\varphi^{-1}(0)$ est fermé et que~$L^2(\Omega)$ étant un
Hilbert, c'est également un Hilbert pour la norme de~$L^2(\Omega)$.

\medskip
\item[$a$ est continue]
\begin{equation}a(p,q)=\dint_\Omega p\cdot q\le |p|_0 |q|_0 \le \|p\|_{H(\dive;\Omega)} \|q\|_{H(\dive;\Omega)}\end{equation}
et donc~$a$ est continue et~$\|a\|\le 1$.

\medskip
\item[$b$ est continue]
\begin{equation}b(v,q)=\dint_\Omega v \dive q \le |v|_0 |\dive q|_0 \le |v|_0 \|q\|_{H(\dive;\Omega)}\end{equation}
et donc~$b$ est continue et~$\|b\|\le 1$.

\medskip
\item[$f$ est continue]
\begin{equation}\dint_\Omega fv\le |f|_0 |v|_0\end{equation} ce qui implique que la norme de la forme linéaire
est~$\le |f|_0$.

\medskip
\item[$a$ est V-elliptique]
On a:
\begin{equation}V=\left\{ q\in H; \dint_\Omega v \dive q =0,\quad \forall v\in L^2_0(\Omega)\right\}\end{equation}

$q\in H$ implique que~$\dint_\Omega (v+c)\dive q =0$, et donc
$\dint_\Omega w \dive q=0$, $\forall w\in L^2(\Omega)$, ce qui finalement
implique que~$\dive q=0$ dans~$L^2(\Omega)$.

On a donc~$V=\left\{ q\in H; \dive q=0\right\}$.

On a alors:
\begin{equation}a(q,q)=\dint_\Omega |q|^2 = \|q\|^2_{H(\dive;\Omega)}\end{equation} si~$q\in V$, et
donc la V-ellipticité de~$a$ est démontrée.

\medskip
\item[Condition inf-sup]\index{condition inf-sup}
Pour montrer la conditions inf-sup, nous construirons~$q$ à partir d'un~$\varphi$ vérifiant
$\frac{\partial\varphi}{\partial n}=0$ sur~$\Gamma$, i.e.:
\begin{equation}
\forall v\in M,\quad \exists q\in H;\quad \dint_\Omega v \dive q = |v|_0^2 \quad\text{ et }\quad
\|q\|_{H(\dive)}\le \frac1\beta |v|_0
\end{equation}
$q\in H$, $\dive q=v$ et~$q=-\nabla\varphi$, $q\cdot n=0$ donc~$\frac{\partial\varphi}{\partial n}=0$,
et:
\begin{equation}\left\{
\begin{aligned}
-&\Delta \varphi = v\\
&\frac{\partial \varphi}{\partial n}=0
\end{aligned}
\right.
\end{equation}
\end{description}
Comme au paragraphe sur la formulation variationnelle, $\exists!\varphi\in H^1(\Omega)\cap L^2_0(\Omega)$
car~$v\in L^2_0(\Omega)$:
\begin{equation}
|q|_0^2=\dint_\Omega |\nabla \varphi|^2 = |\varphi|_1^2 = \dint_\Omega v\varphi \le |v|_0 |\varphi|_0
\le c(\Omega) |V|_0 |\varphi|_1\end{equation} d'où~$|a|_0\le c(\Omega) |V|_0$. De plus:
\begin{equation}|q|_0^2+|\dive q|_0^2 = |q|_0^2+|v|_0^2 \le (C^2(\Omega)+1)|v|_0^2\end{equation}
d'où:
\begin{equation}\beta = \frac1{\sqrt{1+c^2(\Omega)}}\end{equation}

\medskip
\textcolorblue{On peut donc appliquer le théorème de
Brezzi\index{théorème!de Brezzi}\index[aut]{Brezzi (Franco), 1945-, Italien}
et conclure à l'existence et à l'unicité de la solution.}





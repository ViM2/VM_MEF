\chapter*{Introduction}\phantomsection
En introduction à cette partie, il nous semblait important d'en exposer sa structure, car elle peut sembler un peu décousue à la simple lecture de la table des matières.

\medskip
Après les pré-requis exposés dans les deux premières parties, la partie III va s'ouvrir «tout naturellement» sur une présentation générale de la méthode des éléments finis au chapitre~\ref{Ch-MEF}.

\medskip
Immédiatement après, compte tenu du public visé, le chapitre~\ref{Ch-Model} essayera de mettre en avant les spécificités et surtout la complexité de la mécanique comme champ d'application de la méthode des éléments finis.

Cette mise en garde, au regard de l'expérience, nous semble importante: on a tendance souvent à considérer que la mécanique est quelque chose de très bien maîtrisé, et ce n'est pas le cas. Bien des sujets restent pointus, voire ouverts. Il convient donc de rester prudent, surtout pour ceux ayant une expérience de calcul importante qui les porte parfois à sous estimer les difficultés.

\medskip
Le chapitre~\ref{Ch-Elts}, qui fait suite logiquement en terme de présentation de la méthode des éléments finis, au chapitre~\ref{Ch-MEF}, est souvent le mieux maîtrisé par le public d'ingénieurs, au moins concernant les éléments de Lagrange.
Nous l'avons complété de remarques sur les modèles à plusieurs champs (qui peut faire pour partie écho au chapitre~\ref{Ch-Model}).
Encore une fois, pour le public visé, c'est surtout le paragraphe~\ref{Sec-ValidEF} sur la validation pratique des éléments finis qui aura sans doute le plus de valeur ajoutée. Le contenu de ce paragraphe fait souvent partie des choses oubliées.

\medskip
Comme nous en serons sur des choses un peu oubliées, nous en profiterons au chapitre~\ref{Ch-Amelio} pour continuer dans le même sillon et rappeler quelques méthodes d'amélioration de la performance du calcul. Nous y présentons des choses qui sont utilisées plus ou moins fréquemment par notre public.
Seuls les paragraphes~\ref{Sec-PInv} et~\ref{Sec-Deriv} (sur les méthodes de réanalyse et aux dérivées d'ordre élevé) sont généralement moins bien connus. Nous avons voulu les introduire ici plutôt qu'au chapitre~\ref{Ch-XFEM} car ils sont vraiment en lien avec les préoccupations directes de notre public.

\medskip
Le chapitre~\ref{Ch-mesh} permettra une petite pause en exposant brièvement les stratégies de maillage, et plus particulièrement la triangulation de Delaunay.

\medskip
Pour continuer sur les choses pouvant avoir un réel intérêt pratique pour le public d'ingénieurs mécaniciens (ou acousticiens), nous aborderons au chapitre suivant~\ref{Ch-Homog} les méthodes d'homogénéisation.
Si les méthodes les plus «physiques» sont bien connues, l'approche mathématique (généralisante) est bien souvent quasi totalement inconnue.

\medskip
S'en suivra au chapitre~\ref{Ch-Optim} une introduction à l'optimisation, où nous aurons le loisir de revenir sur les multiplicateurs de Lagrange. Nous y aborderons l'optimisation de forme, dont la version «optimisation topologique» recourra aux techniques d'homogénéisation vues juste avant.

\medskip
À ce niveau du document, il nous semble que nous aurons parcouru bon nombre des applications typiques de la méthode des éléments finis, surtout dans le cadre de la mécanique... mais uniquement sous l'angle statique !

Il sera donc temps d'aborder «le temps», i.e. les problèmes non stationnaires, au chapitre~\ref{Ch-temps}.
La propagation des ondes, quant à elle, ne sera traitée qu'au chapitre~\ref{Ch-ondes}. C'est d'ailleurs dans ce chapitre que seront abordés également les modes propres, dont nous n'aurons pas parlé jusqu'alors (à quelques exception près lors de remarques diverses et variées... mais rien de sérieux). Le chapitre~\ref{Ch-Acous} continuera la spécification en se focalisant sur l'acoustique.

\medskip
Dès lors, on pourra considérer qu'une présentation assez complète de la méthode des éléments finis a été faite. Toutefois, ce qui était encore de l'ordre de la recherche il y a une décennie fait aujourd'hui partie des phénomènes que notre public doit prendre en compte de plus en plus souvent. Ces phénomènes, un peu plus complexes, un peu plus exotiques, sont souvent liés à ce que l'on appelle la ou plutôt les non linéarités. C'est ce qui sera abordé au chapitre~\ref{Ch-NL}.

Parmi toutes les non linéarités, c'est celle des lois de comportement qu'il nous est demandé d'exposer en priorité. Les problèmes de contact ou de grands déplacements, qui seront abordés dans ce chapitre, nous sont bien moins demandés car souvent traités par des personnes très au fait des méthodes et même souvent en lien avec la recherche dans le domaine.

\medskip
Quant au traitement de la modélisation de la rupture en mécanique, le chapitre~\ref{Ch-rupt} y sera consacré. Il s'agit néanmoins d'une affaire de spécialistes, et nous ne ferons qu'aborder le sujet.

\medskip
Le chapitre~\ref{Ch-stocha} présentera brièvement quelques méthodes probabilistes pour la prise en compte des aléas dans les équations considérées. C'est encore une fois sous l'angle de la «mécanique aléatoire» que nous illustrerons cela, en incluant une présentation de l'indice de fiabilité d'une structure.

\medskip
Le chapitre~\ref{Ch-XFEM} permet de mettre l'accent sur le fait que même si la méthode des éléments finis est une méthode extrêmement générale, performante et répandue, elle n'est qu'une méthode numérique parmi une multitude d'autres méthodes.
Ce chapitre n'a pas vocation bien évidemment à approfondir aucune des méthodes abordées, mais uniquement à fournir un petit complément culturel qui nous semblait indispensable dans le monde actuel.

\medskip
Le chapitre~\ref{Ch-Singul} clôt cette partie sur un petit rappel lié aux singularités, juste comme une petite piqûre de rappel sur des problèmes dont les ingénieurs pratiquant le calcul numérique restent assez conscients. Il nous a semblé que passer les singularités sous silence pouvait laisser à penser que ce problème n'était peut-être pas si important, ce qui n'est pas le cas bien évidemment.


%\medskip
%Enfin, à notre grand regret (mais ce document est déjà suffisamment, voire même trop, volumineux) certaines méthodes n'ont pas été abordées et feront l'objet de fascicules séparés.
%Nous pensons par exemple en acoustique aux méthodes de tir de rayons, les méthodes énergétiques simplifiées, les «Wave based methods»... i.e. les méthodes aussi bien temporelles que spectrales.




\chapter{Problèmes physiques: formulations faibles et variationnelles}
\begin{abstract}
Nous allons maintenant reprendre les problèmes types exposés sous forme
forte dans le chapitre sur les équations différentielles et EDP, pour leur appliquer les méthodes
du chapitre précédent.
Nous obtiendrons alors les formulations faibles et variationnelles des mêmes problèmes.
\end{abstract}

\medskip
\section{Phénomènes de propagation et de diffusion}
\medskip
\subsection{Équations de Laplace et Poisson}\index{ED-EDP!de Laplace}\index{ED-EDP!de Poisson}\index[aut]{Laplace (Pierre Simon de -), 1749-1827, Français}\index[aut]{Poisson (Siméon Denis), 1781-1840, Français}\index{laplacien}

Sur ce premier exemple, nous détaillerons plus la démarche et les étapes que dans
la suite des cas présentés.

\medskip
\subsubsection{Laplacien de Dirichlet}\index[aut]{Dirichlet (Johann Peter Gustav Lejeune), 1805-1859, Allemand}\index{ED-EDP!de Laplace}\index{laplacien}\index{ED-EDP!de Poisson}\index{condition aux limites!de Dirichlet}\index[aut]{Poisson (Siméon Denis), 1781-1840, Français}
On considère l'équation de Poisson (Laplace si~$f=0$) avec les conditions aux limites de Dirichlet,
i.e. le problème suivant:
\begin{equation}\left\{\begin{aligned}
-&\Delta u=f &&\text{ dans } \Omega\\
&u=0 &&\text{ sur } \Gamma=\partial\Omega
\end{aligned}
\right.
\end{equation}
Multiplier par une fonction test~$v$ et intégrer par parties (ou utiliser la formule de Green) conduit à:
\begin{equation}
\dint_\Omega \nabla u\cdot\nabla v -\dint_\Gamma \dfrac{\partial u}{\partial n}v =
\dint_\Omega fv
\end{equation}
Comme, rappelons le, le but est de faire en sorte que~$u$ et~$v$ jouent un
rôle symétrique, on veut donc qu'ils aient même régularité et mêmes
conditions aux limites. La régularité nécessaire est~$u$ et~$v$ sont
$H^1(\Omega)$, et la prise en compte des conditions aux limites conduit finalement à chercher
$u$ et~$v$ dans~$H^1_0(\Omega)$.

Le problème est donc:
\begin{equation}
\text{Trouver } u \in H^1_0(\Omega) \text{ tel que }
\forall v\in H^1_0(\Omega),\quad
\dint_\Omega \nabla u\cdot\nabla v = \dint_\Omega fv
\end{equation}
dont le second membre n'a de sens que si~$f\in L^2(\Omega)$, ce qui
sera supposé.
\textcolorgris{Notons que le problème est également bien défini dans le cas
où~$f\in H^{-1}(\Omega)$, car~$v\in H^1_0(\Omega)$.}

Avec cette formulation, le théorème de Lax-Milgram permet de conclure
à l'existence et à l'unicité de la solution.

\medskip\colorgreen
\paragraph{Équivalence des solutions}
Nous avons construit un problème variationnel dont on sait qu'il a une
unique solution.
Une question naturelle est alors de savoir en quel sens on a résolu le problème initial.

Il est facile de voir que l'équation~$-\Delta u = f$ est vérifiée au sens des distributions.
Or~$f\in L^2(\Omega)$, donc~$-\Delta u \in L^2(\Omega)$, et l'égalité a donc lieu presque
partout.
De la même manière, on a~$u = 0$ presque partout sur~$\Gamma$ (pour la mesure
surfacique sur~$\Gamma$).

On comprend pourquoi, parmi différentes formulations faibles, on en retient une
plutôt qu'une autre: afin d'obtenir une formulation variationnelle à partir de laquelle
on raisonne rigoureusement, i.e. existence et unicité du résultat, puis retour au problème
initial.
\colorblack

\medskip\colorgris
\paragraph{Une autre formulation variationnelle}
Afin d'illustrer la remarque précédente, construisons une autre formulation
variationnelle du même problème.
Pour cela, intégrons une nouvelle fois par parties, afin de chercher
une solution encore plus faible.
Le problème devient:
\begin{equation}
\text{Trouver } u \in L^1(\Omega) \text{ tel que }
\forall v\in C_c^\infty(\Omega),\quad
-\dint_\Omega u\Delta v = \dint_\Omega fv
\end{equation}
Tout d'abord, on peut remarquer que l'on ne peut plus appliquer Lax-Milgram.
Ensuite, rappelons une remarque déjà faite:
plus on cherche des solutions en un sens faible, plus il est facile de prouver l'existence
de solutions, mais plus il est difficile d'obtenir l'unicité de la solution.

En travaillant dans~$L^1(\Omega)$, on ne pourrait pas, par exemple, donner de sens
à~$u = 0$ sur~$\Gamma$.
\colorblack

\medskip
Comme nous l'avons vu au chapitre précédent, \textcolorred{l'intérêt des problèmes
symétriques, c'est qu'ils peuvent s'interpréter en terme de minimisation.}

La solution du problème initial est également la fonction~$u$ qui réalise
le minimum de:
\begin{equation}
\min_{u\in H^1_0(\Omega)} \dfrac12\dint_\Omega |\nabla u|^2-\dint_\Omega fu
\end{equation}

\medskip
\subsubsection{Laplacien de Dirichlet relevé}\index[aut]{Dirichlet (Johann Peter Gustav Lejeune), 1805-1859, Allemand}\index{ED-EDP!de Laplace}\index{laplacien}\index[aut]{Poisson (Siméon Denis), 1781-1840, Français}\index{ED-EDP!de Poisson}\index{condition aux limites!de Dirichlet}\index[aut]{Laplace (Pierre Simon de -), 1749-1827, Français}
On considère l'équation de Laplace-Poisson avec les conditions aux limites suivantes:
\begin{equation}\left\{\begin{aligned}
-&\Delta u=f &&\text{ dans } \Omega\\
&u=g &&\text{ sur } \Gamma=\partial\Omega
\end{aligned}
\right.
\end{equation}
où~$g\in H^{1/2}(\Gamma)$ est une fonction donnée.

\medskip
L'idée est évidemment de se ramener au cas précédent en réalisant un
\textcolorblue{relèvement d'espace}.

Pour cela, on considère la fonction~$v = u - \overline{u}$ où
$\overline{u}\in H^1(\Omega)$ est un relèvement de la condition
au bord, i.e. tel que~$\overline{u}=g$ sur~$\Gamma$.

On utilise ensuite la surjectivité de l'application trace, $\gamma_0$ (ce qui
devrait malgré tout rester compréhensible bien que la trace ait été définie
de manière <<~un peu légère~>> au chapitre sur les espaces de
Sobolev).

\medskip
La formulation variationnelle du problème est:
\begin{equation}
\text{Trouver } u \in \left\{ v\in H^1(\Omega), v=g \text{ sur } \Gamma\right\}
 \text{ tel que }
\forall v\in H^1_0(\Omega),
\quad\dint_\Omega \nabla u\cdot\nabla v = \dint_\Omega fv
\end{equation}

\medskip
\subsubsection{Laplacien de Neumann}\index{ED-EDP!de Laplace}\index{laplacien}\index[aut]{Poisson (Siméon Denis), 1781-1840, Français}\index{ED-EDP!de Poisson}\index[aut]{Laplace (Pierre Simon de -), 1749-1827, Français}\index[aut]{Neumann (Carl Gottfried), 1832-1925, Allemand}\index{condition aux limites!de Neumann}
On considère l'équation de Laplace-Poisson avec les conditions aux limites de Neumann,
i.e. le problème suivant:
\begin{equation}\left\{\begin{aligned}
-&\Delta u=f &&\text{ dans } \Omega\\
&\dfrac{\partial u}{\partial n}=g &&\text{ sur } \Gamma=\partial\Omega
\end{aligned}
\right.
\end{equation}

\medskip
On obtient immédiatement la formulation variationnelle suivante:
\begin{equation}
\text{Trouver } u \in H^1(\Omega) \text{ tel que }
\forall v\in H^1(\Omega),\quad \dint_\Omega \nabla u\cdot\nabla v = \dint_\Omega fv + \dint_\Gamma gv
\end{equation}
qui nécessite~$f\in L^2(\Omega)$ et~$g\in L^2(\Gamma)$.

\medskip
On peut remarquer qu'ici une simple intégration par parties a suffit à intégrer les conditions aux
limites dans la formulation variationnelle, i.e. les conditions aux limites n'ont pas eu à être introduites
dans l'espace fonctionnel pour~$u$, contrairement au cas du Laplacien de Dirichlet
(en même temps, demander à une fonction de~$H^1(\Omega)$ d'avoir une dérivée
normale égale à une fonction~$g$ au bord n'a pas de sens).

Lorsque les conditions aux limites s'introduisent d'elles-mêmes comme dans
le cas du Laplacien de Neumann, on dit que l'on tient compte des conditions aux limites de
\textcolorblue{manière naturelle}. Lorsqu'il faut les inclure, on dit qu'on en tient
compte de \textcolorblue{manière essentielle}.\index{condition aux limites!essentielle}\index{condition aux limites!naturelle}

\medskip
Il faut remarquer que le problème variationnel ne peut avoir de solution que
si~$f$ et~$g$ vérifient une \textcolorblue{condition de compatibilité}, i.e. si:
\begin{equation}
\dint_\Omega f + \dint_\Gamma g = 0
\end{equation}

\medskip
Il faut aussi remarquer que si~$u$ est une solution du problème, alors~$u+c$,
où~$c$ est une constante, est également solution.
On ne peut donc pas espérer prouver qu'il existe une unique solution
au problème sans restreindre l'espace fonctionnel pour~$u$.
Pour éliminer l'indétermination due à cette constante additive, on introduit, par exemple, $V = \{u\in H^1(\Omega), \int_\Omega u=0\}$
et on pose le problème sous la forme:
\begin{equation}
\text{Trouver } u \in V \text{ tel que }
\forall v\in V,\quad \dint_\Omega \nabla u\cdot\nabla v = \dint_\Omega fv + \dint_\Gamma gv
\end{equation}

En utilisant l'inégalité de Poincaré-Wirtinger, on montre que le problème est bien
coercif et donc qu'il admet une unique solution.

\medskip
Remarquons que si la condition de compatibilité n'est pas vérifiée,
alors la solution de la dernière formulation résout bien le problème initial avec
un~$f$ ou un~$g$ modifié d'une constante additive, de manière à vérifier
la condition de compatibilité.

\medskip
Il y aurait lieu à quelques discussions sur l'interprétation des résultats,
mais celle-ci serait beaucoup plus abstraite que dans le cas du Laplacien de Dirichlet.
Le lecteur désireux de rentrer (à raison) dans ce genre de détail trouvera
aisément un cours de M2 sur le sujet.
Cela dépasse le cadre que nous nous sommes fixé dans ce document.

\medskip
\subsubsection{Laplacien avec conditions mixtes}\index{ED-EDP!de Laplace}\index{laplacien}\index{ED-EDP!de Poisson}\index{condition aux limites!mêlée}\index[aut]{Laplace (Pierre Simon de -), 1749-1827, Français}\index[aut]{Poisson (Siméon Denis), 1781-1840, Français}
On considère l'équation de Laplace-Poisson avec les conditions aux limites mixtes suivantes:
\begin{equation}\left\{\begin{aligned}
-&\Delta u=f &&\text{ dans } \Omega\\
&\dfrac{\partial u}{\partial n}+\alpha u=0 &&\text{ sur } \Gamma=\partial\Omega
\end{aligned}
\right.
\end{equation}

\medskip
On obtient la formulation variationnelle:
\begin{equation}
\text{Trouver } u \in H^1(\Omega), \text{ tel que }
\forall v\in H^1(\Omega),\quad \dint_\Omega \nabla u\cdot\nabla v +\dint_\Gamma \alpha u v = \dint_\Omega fv + \dint_\Gamma gv
\end{equation}

\medskip
\subsection{Équation d'onde}
Rappelons que l'équation des ondes correspond à un problème d'évolution
hyperbolique linéaire.
Différentes conditions aux limites peuvent être imposées.
Dans le cadre de ce document, nous nous intéresserons plus particulièrement
à l'acoustique.
Ce paragraphe est donc traité un peu plus loin dans le document.

\medskip
\subsection{Équation de la chaleur}\index{ED-EDP!de la chaleur}\index{condition aux limites!de Dirichlet}
Après avoir présenté le problème elliptique du Laplacien, intéressons-nous
maintenant au cas parabolique de l'équation de la chaleur.
Pour cela, considérons le problème suivant:
Trouver~$u(x,t)$ tel que:
\begin{equation}
\left\{
\begin{aligned}
&\partial_t u -\Delta u = f &&\text{ dans } [0,T]\times\Omega\\
&u(t,\cdot) = 0 &&\text{ sur } [0,T]\times\Gamma\\
&u(0,\cdot)=u_0 &&\text{ sur } \Omega
\end{aligned}
\right.
\end{equation}
\medskip
On opère comme précédemment.
Toutefois, nous faisons jouer à la variable de temps~$t$ et la variable d'espace~$x$
des rôles différents: on considère la fonction~$u(t, x)$ comme une fonction
du temps, à valeur dans un espace fonctionnel en~$x$, et on prend comme fonctions
tests des fonctions qui dépendent seulement de la variable~$x$.
On obtient alors facilement la formulation variationnelle suivante:
\begin{equation}
\text{Trouver } u \text{ tel que } \forall v\in H^1_0(\Omega),\quad
\dfrac{\mathrm d}{\mathrm dt}\dint_\Omega uv+\dint_\Omega \nabla u\cdot\nabla v=\dint_\Omega fv
\end{equation}
Pour donner un sens à cette formulation variationnelle, une certaine régularité de~$u$ est requise:
\begin{equation}
u\in V = C^0\left([0,T],L^2(\Omega)\right)\cap L^2\left([0,T],H^1_0(\Omega)\right)
\end{equation}
L'espace~$C^0\left([0,T],L^2(\Omega)\right)$ est l'espace des fonctions~$v(t,x)$ telles
que pour tout~$t$, $v(t,\cdot)$ est une fonction de~$L^2(\Omega)$;
l'espace~$L^2\left([0,T],H^1_0(\Omega)\right)$ est l'espace des fonctions~$v(t,x)$ telles
que pour presque tout~$t$, $v(t,\cdot)$ est une fonction de~$H^1_0(\Omega)$ et
$\int_0^T \|v\|^2_{H^1_0(\Omega)}<\infty$. Le premier est un Banach, le second, un Hilbert.

\medskip
La formulation variationnelle est donc:
\begin{equation}
\text{Trouver } u\in V \text{ tel que } u(0,\cdot)=u_0 \text{ et tel que } \forall v\in H^1_0(\Omega),\quad
\dfrac{\mathrm d}{\mathrm dt}\dint_\Omega uv+\dint_\Omega \nabla u\cdot\nabla v=\dint_\Omega fv
\end{equation}
\medskip
Pour résoudre le problème, il faut utiliser une méthode numérique. On montre l'existence puis l'unicité par les estimateurs à priori.
Nous détaillerons cela plus loin.

\medskip
\section{Mécanique des fluides}
Pour la présentation des formulations faibles, nous commencerons par le cas de
l'équation de Stokes avant de passer à celui de l'équation de Navier-Stokes (afin
de nous permettre d'introduire les notations progressivement).


\medskip
\subsection{Équation de Stokes}\index{condition aux limites!de Dirichlet}\index[aut]{Dirichlet (Johann Peter Gustav Lejeune), 1805-1859, Allemand}\index{ED-EDP!de Stokes}\index[aut]{Stokes (George Gabriel), 1819-1903, Anglais}
Si l'on considère la problème de Stokes pour un fluide incompressible avec condition de Dirichlet, il
nous faut résoudre le système:
\begin{equation}\left\{
\begin{aligned}
-&\eta \Delta u + \nabla p = \rho f && \text{ dans } \Omega\\
&\dive u = 0&& \text{ dans } \Omega\\
&u =0&& \text{ sur } \Gamma=\partial\Omega
\end{aligned}\right.
\end{equation}
La formulation mixte qui en découle est:
\begin{equation}
\eta\dint_\Omega \nabla u: \nabla v %d\Omega
- \dint_\Omega p \dive v %d\Omega
= \dint_\Omega f\cdot v %d\Omega
\end{equation}
On introduit les formes bilinéaires:
\begin{equation}
\begin{aligned}
&a: H^1_0(\Omega)^n\times H^1_0(\Omega)^n \rightarrow \RR \\
&(u,v) \mapsto a(u,v)=\eta\dint_\Omega \nabla u: \nabla v
\end{aligned}
\end{equation}
et
\begin{equation}
\begin{aligned}
&b: H^1_0(\Omega)^n\times L^2(\Omega) \rightarrow \RR\\
&(u,q) \mapsto b(u,q)=-\dint_\Omega (\dive v) q
\end{aligned}
\end{equation}
\medskip
On introduit également \textcolorblue{l'espace~$L^2_0(\Omega)$ des fonction~$L^2(\Omega)$
à moyenne nulle}:
\begin{equation}L^2_0(\Omega)=\left\{ q\in L^2(\Omega),\quad \dint_\Omega q %d\Omega
= 0\right\}\end{equation}
\medskip
La \textcolorblue{formulation variationnelle mixte (vitesse/pression)} est alors:
Trouver~$(u,p)\in H^1_0(\Omega)\times L^2_0(\Omega)$ tels que:
\begin{equation}\left\{
\begin{array}{rll}
a(u,v)+b(v,p) &= (f,v) & \forall v\in H^1_0(\Omega)^n \\
b(u,q) &=0 &\forall q \in L^2_0(\Omega)
\end{array}\right.
\end{equation}
Notons que dans le seconde équation, la fonction test~$q$ pourrait de manière équivalente
être cherchée dans~$L^2(\Omega)$ (pas besoin de la moyenne nulle).

Par contre, il est important de chercher~$p \in L^2_0(\Omega)$. C'est cette condition
de pression nulle qui assure l'unicité de la pression (comme dans le cas précédent du
Laplacien de Neumann).

\medskip
Il est possible de formuler ce problème de manière non mixte, i.e. sous une forme
compatible avec Lax-Milgram.
Considérons~$V=\left\{ u\in H^1_0(\Omega)^n / \dive u=0 \text{ dans } \Omega\right\}$.
Alors~$u$ (la vitesse) est solution du problème:
\begin{center}
Trouver~$u\in V$ tel que~$a(u,v)=(f,v)$, $\forall v\in V$.
\end{center}
D'ailleurs c'est cette formulation qui permet ($a$ est continue et coercitive sur~$V$) de
conclure à l'existence et à l'unicité de~$u$.
De là, il ne reste plus qu'à démontrer la même chose pour~$p$ dans la formulation mixte.

\medskip
\subsection{Équation de Navier-Stokes}\index{condition aux limites!de Dirichlet}\index{ED-EDP!de Navier-Stokes}\index[aut]{Stokes (George Gabriel), 1819-1903, Anglais}\index[aut]{Navier (Claude Louis Marie Henri), 1785-1836, Français}\index[aut]{Dirichlet (Johann Peter Gustav Lejeune), 1805-1859, Allemand}

Considérons le problème de Navier-Stokes, pour un fluide incompressible, et soumis
à des conditions initiales en temps et sur le contour~$\Gamma=\partial\Omega$:
\begin{equation}\left\{
\begin{aligned}
&\dfrac{\partial u}{\partial t} + (u\cdot\nabla) u + \eta\Delta u + \nabla p = \rho f && \text{ dans } \Omega\times\RR^+\\
&\dive u = 0&& \text{ dans } \Omega\times\RR^+\\
&u=g &&\text{ sur } \Gamma\times\RR^+\\
&u(\cdot,0) =u_0 &&\text{ dans }\Omega
\end{aligned}\right.
\end{equation}

\textcolorblue{La fonction~$g=g(x,t)$ dans la condition de type Dirichlet sur le bord du domaine
doit être à flux nul sur~$\Gamma$}, i.e. on suppose que:
\begin{equation}
\dint_\Gamma g\cdot n %d\Gamma
= 0
\end{equation}
avec~$n$ la normale extérieure à~$\Gamma$.
Cette condition est \textcolorred{nécessaire} pour que le problème de Navier-Stokes admette une
solution.

\medskip
Introduisons également \textcolorblue{les espaces de fonctions} à valeurs dans un espace
de Banach~$B$. Pour~$q\ge1$, on a:
\begin{equation}
L^q(0,T;B) =\left\{ v: [0,T]\rightarrow B; v \text{ est mesurable et }
\dint_0^T \|v(t)\|^q_B \mathrm dt<+\infty\right\}
\end{equation}
Ces espaces sont munis de la norme:
\begin{equation}
\|u\|_{L^q(0,T;B)} = \left(\dint_\Omega\|u(t)\|^q_B \mathrm dt\right)^{1/q}
\end{equation}
Notons que ces espaces ont été <<~présentés~>> au paragraphe précédent
sur l'équation de la chaleur. La notation était un peu différente mais il s'agit bien
des mêmes espaces. Nous avons voulu profiter de ces exemples pour introduire
ces deux notations, différentes mais désignant bien la même chose.

\medskip
Posons~$X=H^1_0(\Omega)^n$ et~$Y=L^2_0(\Omega)^n$.
\medskip
La \textcolorblue{formulation variationnelle mixte des équations de Navier-Stokes} s'écrit: Soit~$f\in L^2(0,T;L^2(\Omega)^n)$ et~$u_0\in L^2(\Omega)^n$.
Trouver:
\begin{equation}u\in W(0,T)=\left\{v\in L^2(0,T;X), \dfrac{\mathrm dv}{\mathrm dt}\in L^2(0,T;X') \right\}\end{equation} et
$p\in L^2(0,T;Y)$, tels que p.p.~$t\in(0,T)$, on a:
\begin{equation}\left\{
\begin{aligned}
&\langle\dfrac{\mathrm du}{\mathrm dt}(t),v\rangle_{X',X}+a(u(t),v)+c(u(t),u(t),v)+b(v,p(t)) = (f(t),v) &&
\forall v\in X\\
&b(u(t),q) =0 &&\forall q \in Y\\
&u(0)=u_0
\end{aligned}\right.
\end{equation}
où~$c: X\times X\times X\rightarrow\RR$ est la forme trilinéaire définie par:
\begin{equation}
c(w,z,v) = \dint_\Omega [(w\cdot\nabla)z]\cdot v %d\Omega
=\dsum_{i,j=1}^n \dint_\Omega w_j\dfrac{\partial z_i}{\partial x_j} v_i %d\Omega
\end{equation}

\medskip
Cette forme trilinéaire a certaines propriétés, par exemple lorsque l'on permute
des termes où lorsque l'on considère que certaines variables sont à divergence nulle...

\medskip
Pour des raisons de stabilité, la forme~$c$ est remplacée par le forme~$\tilde{c}$ définie par:
\begin{equation}
\tilde{c} = \dfrac12 \left(c(w,z,v)-c(w,v,z)\right)
\end{equation}
qui est antisymétrique et possède par conséquent la propriété~$\tilde{c}(w,z,z)=0$,
$\forall w,x,z \in X$.

Nous ne rentrons pas plus avant dans la méthode, car il faudrait alors parler dès à présent
du schéma numérique associé, ce qui sera fait plus tard.

\medskip
\subsection{Équation d'Euler}\index{ED-EDP!d'Euler}\index[aut]{Euler (Leonhard Paul), 1707-1783, Suisse}
Nous avons déjà vu que sans le cas d'un nombre de Reynolds\index[aut]{Reynolds (Osborne), 1842-1912, Irlandais}
élevé, alors le terme de convection non-linéaire~$(u.\nabla)u$ devient prépondérant,
et on obtient alors l'equation d'Euler.
Pour un fluide incompressible, le problème à résoudre est donc:
\begin{equation}\left\{
\begin{aligned}
&\dfrac{\partial u}{\partial t} + (u\cdot\nabla) u + \nabla p = \rho f && \text{ dans } \Omega\times\RR^+\\
&\dive u = 0&& \text{ dans } \Omega\times\RR^+
\end{aligned}\right.
\end{equation}
On considérera qu'il s'agit d'un cas simplifié de Navier-Stokes, et on le traitera
comme tel.

\medskip
\section{Équations de la mécanique des milieux continus des solides}
\medskip
\subsection{Formulation générale}
La formulation la plus générale du problème est:\index{ED-EDP!relation fondamentale de la dynamique}\index{loi de comportement}\index{condition aux limites!de Dirichlet}\index{condition aux limites!de Neumann}\index[aut]{Dirichlet (Johann Peter Gustav Lejeune), 1805-1859, Allemand}\index[aut]{Neumann (Carl Gottfried), 1832-1925, Allemand}
\begin{equation}
\left\{
\begin{aligned}
&\dive\sigma + f = \rho \ddot{u} + \mu \dot{u} && \text{ dans } \Omega && \text{ équation de la dynamique}\\
&\sigma=D(\varepsilon-\varepsilon_{th}) && \text{ dans } \Omega && \text{ loi de comportement}\\
\end{aligned}\right.
\end{equation}
à laquelle il est nécessaire d'ajouter les conditions aux limites suivantes:
\begin{equation}
\left\{
\begin{aligned}
&u=\overline{u} && \text{ sur } \Gamma_D && \text{ déplacement imposé}\\
&\sigma\cdot n(x)=g_N && \text{ sur } \Gamma_N && \text{ condition sur le bord}\\
&\dot{u}(t=0) = \dot{u}_0&& && \text{ condition initiale en vitesse}\\
&u(t=0) = u_0 && && \text{ condition initiale en déplacement}\\
\end{aligned}
\right.
\end{equation}
avec comme d'habitude, $\Gamma_D$ et~$\Gamma_N$ une partition de la
frontière~$\Gamma=\partial\Omega$ du domaine~$\Omega$.

\medskip
Pour obtenir une formulation faible, on applique les mêmes recettes
que ci-dessus. Dans un premier temps, on ne traite que l'équation de
la dynamique, la loi de comportement n'étant là finalement que pour
<<~simplifier~>> l'équation en permettant de lier des variables entre elles.
Ainsi, la formulation faible générale est:
\begin{equation}
\dint_\Omega \rho v\ddot{u} + \dint_\Omega \mu v\dot{u}
+\dint_\Omega \varepsilon:\sigma - \dint_\Omega fv
-\dint_{\Gamma_N} g_N v =0, \quad \forall v \text{ tel que } v=\overline{u} \text{ sur } \Gamma_D
\end{equation}
Nous sommes restés très prudents sur l'appartenance des différentes grandeurs
à des espaces, car cela dépend du choix des variables.

\medskip
\subsection{Choix des variables}\label{Sec-form}

Dans la formulation variationnelle présentée, on ne manquera pas de remarquer
que les variables sont: les déplacements (et leurs dérivées temporelles: vitesse et accélération),
les déformations et les contraintes.

Il est possible de choisir de conserver toutes ces variables, i.e. d'avoir un problème
dépendant de trois champs inconnus:~$u$, $\varepsilon$ et~$\sigma$.
Une telle formulation est qualifiée de \textcolorblue{mixte}.
On qualifie d'ailleurs de mixte, toute formulation ayant plus d'un seul champ inconnu.

Comme on connaît des relations entre les différents champs (relation déplacement /
déformation et relation déformation / contrainte), il est possible
d'obtenir des formulations ayant deux ou même un seul champ inconnu, comme
nous l'avons déjà vu au chapitre précédent (théorèmes de Brezzi).\index[aut]{Brezzi (Franco), 1945-, Italien}

\medskip
\subsubsection{Formulation classique en déplacements}
L'intérêt de réduire le nombre de champ inconnu est évident.
Faut-il donc opter définitivement pour une formulation à un seul champ?
Ce n'est pas évident. En effet, utiliser une formulation à~$k$ champs parmi~$n$
se traduira par n'obtenir que ces~$k$ champs comme solution directe du
système. Les~$n-k$ champs manquant doivent alors être obtenus à partir des
$k$ champs choisis... et cela peut poser certains problèmes, dont nous parlerons plus tard.

Il est à noter que la formulation dite \textcolorblue{en déplacements}, également
qualifiée de \textcolorblue{classique}, est la plus utilisée. Comme son nom
l'indique, elle ne comporte que le champ de déplacement comme inconnu.
Il suffit de considérer, dans la formulation précédente, que les contraintes
sont obtenues à partir des déformations, via la loi de comportement, et que
les déformations sont obtenues à partir des déplacements par la relation
idoine (en petits ou grands déplacements).

\medskip
Dans la formulation en déplacement, on suppose que les relations liant les
déformations aux déplacements (opérateur de dérivation en petites ou grandes
déformations) et les contraintes aux déplacements (en fait les contraintes aux
déformations via la loi de comportement, puis les déformations aux déplacements
comme ci-avant) sont connues et vérifiées:
\begin{equation}
\dint_\Omega \rho v\ddot{u} + \dint_\Omega \mu v\dot{u}
+\dint_\Omega \varepsilon(v):\sigma(u) - \dint_\Omega fv
-\dint_{\Gamma_N} g_N v =0, \quad \forall v \text{ tel que } v=\overline{u} \text{ sur } \Gamma_D
\end{equation}
\colorgris
De manière plus explicite, si on note~$\varepsilon=\mathcal{L}u$ la relation entre
déformations et déplacements et~$\sigma = H\varepsilon = H\mathcal{L}u$ la relation entre
contraintes et déplacement, alors il vient:
\begin{equation}
\dint_\Omega \rho v\ddot{u} + \dint_\Omega \mu v\dot{u}
+\dint_\Omega \mathcal{L}v:H\mathcal{L}u - \dint_\Omega fv
-\dint_{\Gamma_N} g_N v =0, \quad \forall v \text{ tel que } v=\overline{u} \text{ sur } \Gamma_D
\end{equation}
où n'apparaissent effectivement plus que les déplacements comme inconnues.\colorblack

\textcolorgreen{On se contente toutefois généralement d'écrire~$\varepsilon(u)$ et~$\sigma(u)$ pour
dire que l'inconnue est bien~$u$, ce qui évite d'alourdir l'écriture.}

\medskip
Cette formulation <<~en déplacement~>> correspond au problème de minimisation
de la \textcolorblue{fonctionnelle de l'énergie potentielle totale}\index{fonctionnelle!de l'énergie potentielle totale}
que l'on appelle également \textcolorblue{fonctionnelle primale}\index{fonctionnelle!primale}:
\begin{equation}
J(u)=
\frac12 \dint_\Omega \varepsilon(v):\sigma(u) - \dint_\Omega fv -
\dint_{\Gamma_N} g_N v
\end{equation}
où le champ de déplacements doit vérifier~$v=\overline{u}$ sur~$\Gamma_D$.
Les relations entre déformations et déplacements et entre contraintes et déplacements
sont supposées vérifiées de manière implicite dans~$\Omega$.

\medskip
\subsubsection{Illustration: formulation en déplacements de la statique des matériaux isotropes}\index{ED-EDP!relation fondamentale de la dynamique}\index{loi de comportement}\index{condition aux limites!de Dirichlet}\index{condition aux limites!de Neumann}\index[aut]{Dirichlet (Johann Peter Gustav Lejeune), 1805-1859, Allemand}\index[aut]{Neumann (Carl Gottfried), 1832-1925, Allemand}\index{loi de comportement!Hooke}\index[aut]{Hooke (Robert), 1635-1703, Anglais}

En se plaçant dans le cas statique, et en écrivant <<~en déplacements~>>, i.e.
en supprimant le champ de contrainte des inconnues (et en utilisant, au passage,
la symétrie de~$\sigma$), la formulation variationnelle précédente devient:
\begin{equation}
\dint_\Omega H_{ijkl}\varepsilon_{ij}(u)\varepsilon_{kl}(v) = \dint_\Omega fv +
\dint_{\Gamma_N} g_N v, \quad \forall v\in H^1_D(\Omega)
\end{equation}
\textcolorgris{Notons que pour prouver la coercitivité de la forme bilinéaire,
il faut recourir à l'inégalité de Korn, qui a été présentée précédemment à la
fin de la partie 1.}

\medskip
Si en plus le matériau est isotrope, cette expression se simplifie en:
\begin{equation}
\dint_\Omega \lambda \dive(u)\dive(v)+2\mu\varepsilon(u)\cdot\varepsilon(v) = \dint_\Omega fv +
\dint_{\Gamma_N} g_N v, \quad \forall v\in H^1_D(\Omega)
\end{equation}
que l'on peut voir comme un problème de minimisation de la fonctionnelle:
\begin{equation}
J(u)=
\dint_\Omega \lambda (\dive u)^2+2\mu\dsum_{ij}\varepsilon_{ij}(u)^2 - \dint_\Omega fv -
\dint_{\Gamma_N} g_N v
\end{equation}
\medskip
D'une manière générale, toute formulation faible, quelque soit le nombre
de champs inconnus, s'écrit en mécanique sous la forme de minimisation d'une
fonctionnelle, dont on sait trouver la signification physique.

\medskip
\subsubsection{Formulations mixte et hybride}\label{Sec-MH}
La formulation la plus générale est celle comportant les trois champs de déplacements
$u$, de déformations~$\varepsilon$, et de contraintes~$\sigma$ comme inconnues.
Cette formulation mixte est connue sous le nom de \textcolorblue{principe de Hu-Washizu} (1975).\index[aut]{Hu (Haichang), 1928-2011, Chinois}\index[aut]{Washizu (Kyuichiro), 1921-1981, Japonais}
Sous forme variationnelle, elle s'écrit:\index{fonctionnelle!de Hu-Washizu}
\begin{equation}
J(u,\varepsilon,\sigma) =
\frac12 \dint_\Omega \varepsilon:\sigma
+\dint_\Omega (\nabla u) \sigma
- \dint_\Omega f v
- \dint_{\Gamma_N} g_N v
- \dint_{\Gamma_D} (u-\overline{u}) (\sigma(v)\cdot n)
\end{equation}
et ne comporte aucune condition (puisqu'elles sont toutes formellement présente dans
la formulation).

\medskip
Il est possible d'obtenir une formulation n'ayant que le champ de contraintes
comme inconnue. Il s'agit de la \textcolorblue{fonctionnelle duale}:\index{fonctionnelle!duale}
\begin{equation}
J(\sigma) =
-\frac12 \dint_\Omega \sigma S \sigma
-\dint_\Omega \sigma(v) S\sigma_0
+ \dint_{\Gamma_D} \overline{u} \sigma(v)\cdot n
\end{equation}
où le champ de contrainte doit vérifier les conditions aux limites~$\dive \sigma + f =0$ sur~$\Omega$
et~$\sigma n = g_N$ sur~$\Gamma_N$.
$S$ désigne la loi de comportement donnée sous forme inverse (i.e. sous forme de
souplesse, i.e.~$S=H^{-1}$).

\medskip
L'utilisation de multiplicateurs de Lagrange permet de construire des fonctionnelles multi-champs.
La \textcolorblue{fonctionnelle de l'énergie complémentaire}\index{fonctionnelle!de l'énergie complémentaire} est la suivante:
\begin{equation}
J(\sigma,\lambda,\mu) =
-\frac12\dint_\Omega \sigma S \sigma
-\dint_\Omega \lambda (\dive \sigma + f)
+ \dint_{\Gamma_D} \overline{u}\sigma\cdot n
+ \dint_{\Gamma_N} \mu (\sigma\cdot n -g_N)
\end{equation}
où~$\lambda$ et~$\mu$ sont les multiplicateurs de Lagrange.\index{multiplicateurs de Lagrange}\index[aut]{Lagrange (Joseph Louis, comte de -), 1736-1813, Italien}
\medskip
Les multiplicateurs de Lagrange présentés ci-dessus ont une signification physique
(que l'on trouve en écrivant la stationnarité de la fonctionnelle).
En remplaçant les multiplicateurs par les quantités qu'ils représentent, i.e. en remplaçant
$\lambda$ par~$u$ dans~$\Omega$ et~$\mu$ par~$u$ sur~$\Gamma_N$, on obtient la
\textcolorblue{fonctionnelle d'Hellinger-Reissner}\index{fonctionnelle!d'Hellinger-Reissner}\index[aut]{Hellinger (Ernst David), 1883-1950, Allemand}\index[aut]{Reissner (Max Erich, dit Eric), 1913-1996, Américain}
sous sa première forme:
\begin{equation}
J(u,\sigma) =
-\frac12 \dint_\Omega \sigma S \sigma
-\dint_\Omega (\dive \sigma+f) u
+ \dint_{\Gamma_D} \overline{u}\sigma\cdot n
- \dint_{\Gamma_N} (\sigma\cdot n-g_N) v
\end{equation}
En notant:
\begin{align} 
&a(\sigma,\sigma)=-\dint_\Omega \sigma S\sigma\\
&b(\sigma,u)=-\dint_\Omega \dive\sigma u + \dint_{\Gamma_D} \overline{u}\sigma\cdot n + \dint_{\Gamma_N} u\sigma\cdot n\\
&L(u)=\dint_\Omega fu + \dint_{\Gamma_N} g_N v\end{align}
alors, le première fonctionnelle de Reissner se met sous la forme:
\begin{equation} J(\sigma,u) = \frac12a(\sigma,\sigma)+b(\sigma_u)-L(u)\end{equation}
\medskip
En effectuant une intégration par parties du terme mixte~$b(\sigma,u)$, on obtient la
\textcolorblue{fonctionnelle d'Hellinger-Reissner}\index{fonctionnelle!d'Hellinger-Reissner}\index[aut]{Hellinger (Ernst David), 1883-1950, Allemand}\index[aut]{Reissner (Max Erich, dit Eric), 1913-1996, Américain}
sous sa deuxième forme:
\begin{equation}
\label{Eq-HR2}
J(u,\sigma) =
-\frac12 \dint_\Omega \sigma S \sigma
+\dint_\Omega \sigma:\varepsilon
- \dint_\Omega f v
- \dint_{\Gamma_N} g_N v
\end{equation}
où le champ de déplacement doit vérifier~$v=\overline{u}$ sur~$\Gamma_D$,
et où la relation entre déformations et déplacement est supposée vérifiée dans~$\Omega$.

Cette fonctionnelle d'Hellinger-Reissner est également appelée \textcolorblue{fonctionnelle mixte}.\index{fonctionnelle!mixte}
\medskip
L'adjectif \textcolorblue{hybride} peut être adjoint à l'une quelconque des formulations
précédentes si le ou les champs sont interpolés d'une part sur tout le domaine~$\Omega$
et d'autre part sur tout ou partie du contour~$\Gamma$ de façon indépendante.

Il est possible de modifier la fonctionnelle de l'énergie complémentaire de plusieurs manière.
Nous ne présentons que celle conduisant à ce que l'on appelle souvent \textcolorblue{fonctionnelle hybride}\index{fonctionnelle!hybride} et
qui est la \textcolorblue{fonctionnelle de Pian et
Tong}\index{fonctionnelle!de Pian et Tong}\index[aut]{Pian (Theodore H.H.), 1919-2009, Américain}\index[aut]{Tong (Pin), ? , Chinois}
(1978):
\begin{equation}
J(u,\varepsilon) =
-\frac12 \dint_\Omega \sigma S \sigma
+\dint_\Gamma u \sigma(v)\cdot n
- \dint_{\Gamma_N} g_N v
\end{equation}

Sa variation:
\begin{equation}
\delta J(u,\varepsilon) =
-\dint_\Omega \sigma S \sigma
+\dint_\Gamma u \sigma(v)\cdot n + v \sigma(u)\cdot n
- \dint_{\Gamma_N} g_N v
\end{equation}
a l'avantage de ne pas comporter de dérivée.

\medskip
Au paragraphe~\ref{Sec-champs}, nous reprendrons ces formulations pour les exprimer,
peut-être de manière plus <<~explicite~>> pour le lecteur sous forme matricielle.

\medskip
\section{Équations de l'acoustique}\index{ED-EDP!de l'acoustique}
On s'intéresse à la propagation et à la réflexion d'ondes de pression dans un
fluide parfait non pesant.
On supposera le mouvement harmonique autour d'un état moyen (ambiance) au repos
\textcolorgris{(ce qui est le lien <<~physique~>> avec l'introduction de l'espace~$L^2_0(\Omega)$
présenté pour l'équation de Stokes)}.

\medskip
On rappelle que l'équation des ondes acoustiques de célérité~$c$ dans un milieu est:
\begin{equation}
\Delta p' - \dfrac1{c^2} \ddot{p}' = 0
\end{equation}

\medskip
\subsection{Équation de Helmholtz}\index{ED-EDP!de Helmholtz}\index[aut]{Helmholtz (Hermann Ludwig Ferdinand von -), 1821-1894, Allemand}
Il s'agit de la formulation générale d'un problème acoustique linéaire.
Elle est obtenue en cherchant la \textcolorblue{solution harmonique} de l'équation
des ondes, i.e. une solution sous la forme:~$p'(x,y,z,t)=p(x,y,z)\mathrm \mathrm{e}^{i\omega t}$.
\textcolorgris{(Au passage, les physiciens utilisent plutôt~$j$ au lieu de~$i$ pour gagner en clarté sans
doute... comme s'ils passaient leur temps à parler d'intensité!)}

On obtient alors l'\textcolorblue{équation de Helmholtz}:
\begin{equation} \Delta p + k^2 p=0 \end{equation}
où~$k$ est le \textcolorblue{nombre d'onde}:~$k=\dfrac\omega{c}=\dfrac{2\pi}\lambda$.

\medskip
\subsection{Conditions aux limites en acoustique}\index{condition aux limites!de Dirichlet}\index{condition aux limites!de Neumann}\index{condition aux limites!de Robin}\index[aut]{Dirichlet (Johann Peter Gustav Lejeune), 1805-1859, Allemand}\index[aut]{Neumann (Carl Gottfried), 1832-1925, Allemand}\index[aut]{Robin (Victor Gustave), 1855-1897, Français}
Soit~$\Gamma=\partial\Omega$ la frontière du domaine concerné.
On considère que~$\Gamma$ est \textcolorblue{partitionné} en trois zones (i.e. dont
l'union vaut~$\Gamma$ et dont aucune n'intersecte les autres) qui
portent chacune une condition aux limites différente:~$\Gamma_D$ de Dirichlet,
$\Gamma_N$ de Neumann et~$\Gamma_R$ de Robin.
\medskip
On a donc:
\begin{itemize}
  \item Dirichlet:~$p=\overline{p}$ sur~$\Gamma_D$;\\[-2ex]
  \item Neumann:~$\partial_n p = -i\rho\omega \overline{V}_n$ sur~$\Gamma_N$;\\[-2ex]
  \item Robin:~$\partial_n p=-i\rho\omega A_n p$ sur~$\Gamma_R$.
\end{itemize}
\medskip\colorgreen
\paragraph{Condition de Neumann}
$\overline{V_n}$ est la valeur imposée de la composante normale de la vitesse.
Cette condition limite modélise la \textcolorred{vibration} d'un panneau, ce qui en fait la condition la
plus souvent utilisée lorsque l'on a un couplage faible entre la structure et le fluide.
On effectue alors une étude vibratoire de la structure, puis grâce à celle-ci une
étude acoustique du fluide sans se préoccuper des interactions fluide-structure.

\medskip
\paragraph{Condition de Robin}
$A_n$ est appelé coefficient d'admittance, il vaut l'inverse de l'impédance~$Z_n$:
$A_n\cdot Z_n=1$.
Physiquement, ce coefficient représente l'\textcolorred{absorption} de l'onde par le matériau environnant
dont la frontière est modélisée par~$\Gamma_R$.
\colorblack

\medskip
\subsection{Formulation faible}
En notant~$H^1_D(\Omega)=\left\{p\in H^1(\Omega), p=\overline{p} \text{ sur } \Gamma_D\right\}$ et
$H^0_D(\Omega)=\left\{v\in H^1(\Omega), v=0 \text{ sur } \Gamma_D\right\}$, les espaces qui sont
des sous-espaces de~$H^1(\Omega)$, on trouve finalement:
\begin{equation}a(p,\tilde{w})=b(\tilde{w}), \quad \forall w\in H^0_D(\Omega)\end{equation}
avec~$\tilde{\cdot}$ la conjugaison complexe, $a(p,q): H^1_D\times H^1_D \rightarrow \CC$ la forme
sesquilinéaire définie par:
\begin{equation}a(p,q)=\dint_\Omega \partial_{x_i}p\partial_{x_i}\tilde{q}-k^2p\tilde{q} +
\dint_{\Gamma_R} i\rho ckA_n\tilde{q}
\end{equation}
et~$b(a): H^1_D\rightarrow\CC$ définie par:
\begin{equation}b(q)=-\dint_{\Gamma_N} i\rho c k\overline{V_n} q\end{equation}

\medskip
Et comme on peut vérifier que l'on est dans un cas qui va bien, on définit l'énergie
potentielle totale du domaine par:
\begin{equation}W_p=\frac12 a(p,\tilde{p})-b(\tilde{p})\end{equation}
et la résolution du problème devient:
\begin{equation}
\text{Trouver } p\in H^1_D(\Omega) \text{ tel que } W_p \text{ soit minimal}
\end{equation}
ou de manière équivalente:
\begin{equation}
\text{Trouver } p\in H^1_D(\Omega) \text{ tel que } \delta W_p=0,\quad \forall\delta p\in H^1_0(\Omega)
\end{equation} 
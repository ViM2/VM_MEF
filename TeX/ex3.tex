\ifVersionAvecExemplesSepares
  \chapter{[\castem] Un calcul de poutre vu au chapitre~\ref{Ch-Model}}

  Nous allons reprendre pour partie l'exemple de la poutre encastrée traitée au chapitre~\ref{Ch-Model} sous \castem..
\else
  \section{Exemple: retour sur le calcul de poutre du paragraphe~\ref{Sec-champs} avec \castem}

  Nous allons reprendre pour partie l'exemple de la poutre encastrée traitée au paragraphe précédent et présenter
  le listing \castem correspondant.
\fi



\medskip
\ifVersionAvecExemplesSepares
  \section{Modélisation 2D}
\else
  \subsection{Modélisation 2D}
\fi

Nous commençons par définir les données du problème: longueur, largeur, épaisseur, nombre d'éléments
selon chacune de ces directions, et force appliquée:

\colorgris
\begin{Verbatim}[numbers=left,numbersep=3pt]
* DONNEES
* geometrie
long1=22.0;
larg1=8.;
ep1=4;
* maillage
nlong1=22;
nlarg1=8;
* effort
Forc1=-41.;
\end{Verbatim}
\colorblack

\medskip
Puis nous définissons la dimension du problème (modèle tridimensionnel), et le type de découpage:

\colorgris
\begin{Verbatim}[numbers=left,numbersep=3pt,firstnumber=last]
OPTION DIME 3 ELEM QUA4;
\end{Verbatim}
\colorblack

\medskip
Nous définissons les points~$k_i$ (dénommés ainsi pour rappel de la syntaxe Ansys des keypoints \verb|k,i,...|), puis les
lignes~$L_i$, et la surface~$S_1$.

\colorgris
\begin{Verbatim}[numbers=left,numbersep=3pt,firstnumber=last]
k1 = 0. 0. 0.;
k2 = long1 0. 0.;
k3 = long1 larg1 0.;
k4 = 0. larg1 0.;
*
L1 = DROI nlong1 k1 k2;
L2 = DROI nlarg1 k2 k3;
L3 = DROI nlong1 k3 k4;
L4 = DROI nlarg1 k4 k1;
*
S1 = DALLER L1 L2 L3 L4;
\end{Verbatim}
\colorblack

\medskip
Le modèle est un modèle de mécanique élastique isotrope utilisant lélément de coque COQ4 pour le maillage~$S_1$:

\colorgris
\begin{Verbatim}[numbers=left,numbersep=3pt,firstnumber=last]
Model1 = MODL S1 MECANIQUE ELASTIQUE ISOTROPE COQ4;
\end{Verbatim}
\colorblack

\medskip
Enfin on résout le problème après avoir fourni les données matérielles et les conditions aux limites:

\colorgris
\begin{Verbatim}[numbers=left,numbersep=3pt,firstnumber=last]
Mater1 = MATERIAU Model1 YOUNG 70000.0 NU 0.33 RHO 2700.0;
Car1 = CARAC Model1 EPAI ep1;
Mater1 = Mater1 ET Car1;
MR1 = RIGIDITE Model1 Mater1;
CL1 = BLOQ DEPL L4;
CL2 = BLOQ ROTA L4;
FOR1 = FORCE(0. 0. Forc1) k3;
MTOT1 = MR1 ET CL1 ET CL2;
Depl1 = RESOUD MTOT1 FOR1 ;
\end{Verbatim}
\colorblack

\medskip
On peut post-traiter les résultats et les afficher.

Les composantes de~$Sig_1$ dans le repère LOCAL sont:
\begin{itemize}
  \item les efforts normaux: N11, N22;
  \item l'effort de cisaillement plan: N12; 
  \item les moments de flexion: M11, M22;
  \item le moment de cisaillement plan: M12;
  \item les efforts tranchants V1, V2.
\end{itemize}

Les composantes de~$Eps_1$ dans le repère LOCAL sont:
\begin{itemize}
  \item les élongations normales dans le plan: EPSS, EPTT;
  \item les cissions dans le plan et transverses: GAST, GASN, GATN;
  \item les courbures: RTSS, RTTT, RTST.
\end{itemize}

\colorgris
\begin{Verbatim}[numbers=left,numbersep=3pt,firstnumber=last]
* DEPLACEMENTS
UZ1 = EXCO 'UZ' depl1;
*TRAC UZ1 S1;
*
* DEFORMEE
def0 = DEFO S1 Depl1 0.0 BLEU;
def1 = DEFO S1 Depl1;
*TRAC (def0 ET def1);
*
* CONTRAINTES: 
Sig1 = SIGM Model1 Mater1 Depl1;
Siigg1 = EXCO M11 Sig1;
TRAC Siigg1 Model1 def1;

* DEFORMATIONS
Eps1 = EPSI Model1 Mater1 Depl1;
Eppss1 = EXCO RTSS Eps1;
*TRAC Eppss1 Model1;

fin;
\end{Verbatim}
\colorblack


\medskip
\ifVersionAvecExemplesSepares
  \section{Modèle 3D}
\else
  \subsection{Modèle 3D}
\fi

Nous allons maintenant construire le modèle tridimensionnel.

Comme nous voulons travailler à l'économie, nous repartons du fichier précédent que nous adaptons.

\colorgris
\begin{Verbatim}[numbers=left,numbersep=3pt]
* DONNEES
* geometrie
long1=22.0;
larg1=8.;
ep1=2;
* maillage
nlong1=22;
nlarg1=8;
nep1=3;
* effort
Forc1=-41.;
\end{Verbatim}
\colorblack

\medskip
Cette fois, nous nous servons de CUB8 au lieu de QUA4.

\colorgris
\begin{Verbatim}[numbers=left,numbersep=3pt,firstnumber=last]
OPTION DIME 3 ELEM CUB8;
*
k1 = 0. 0. 0.;
k2 = long1 0. 0.;
k3 = long1 larg1 0.;
k4 = 0. larg1 0.;
*
L1 = DROI nlong1 k1 k2;
L2 = DROI nlarg1 k2 k3;
L3 = DROI nlong1 k3 k4;
L4 = DROI nlarg1 k4 k1;
*
S1 = DALLER L1 L2 L3 L4;
\end{Verbatim}
\colorblack

\medskip
À partir de la surface~$S_1$, qui est la même que précédemment, nous allons créer le volume~$V_1$ par translation
selon le vecteur~$Vect_1$.

Nous en profitons également pour créer~$Face_1$ sur laquelle porterons les conditions aux limites.
Notons par exemple que la ligne~$L_4$ appartient bien au maillage~$V_1$, puisque ce dernier est construit dessus.
Par contre, la surface~$face_1$ n'appartient pas à~$V_1$, il est donc nécessaire de la lier à~$V_1$ en utilisant
la commande ELIM.

\colorgris
\begin{Verbatim}[numbers=left,numbersep=3pt,firstnumber=last]
Vec1=0. 0. (-1.0*ep1);
V1=S1 VOLU nep1 TRAN Vec1;
Face1 = L4 TRAN nep1 Vec1;
ELIM 0.0000001 V1 Face1;
\end{Verbatim}
\colorblack

\medskip
Cette fois, le modèle correspond au maillage~$V_1$ et utilise l'élément CUB8.

\colorgris
\begin{Verbatim}[numbers=left,numbersep=3pt,firstnumber=last]
Model1 = MODL V1 MECANIQUE ELASTIQUE ISOTROPE CUB8;
\end{Verbatim}
\colorblack

\medskip
On résout, puis on post-traite.

\colorgris
\begin{Verbatim}[numbers=left,numbersep=3pt,firstnumber=last]
Mater1 = MATERIAU Model1 YOUNG 70000.0 NU 0.33 RHO 2700.0;
MR1 = RIGIDITE Model1 Mater1;
CL1 = BLOQ DEPL Face1;
CL2 = BLOQ ROTA Face1;
FOR1 = FORCE(0. 0. Forc1) k3;
MR1 = MR1 ET CL1 ET CL2;
Depl1 = RESOUD MR1 FOR1 ;
*
* DEPLACEMENTS
UZ1 = EXCO 'UZ' depl1;
*TRAC CACH UZ1 V1;
* DEFORMEE
def0 = DEFO V1 Depl1 0.0 BLEU;
def1 = DEFO V1 Depl1;
*TRAC CACH (def0 ET def1);
*
* CONTRAINTES: 
Sig1 = SIGM Model1 Mater1 Depl1;
* Les composantes de Sig1 sont: VONMISES, SMXX, SMYY, SMZZ, SMXY, SMXZ, SMYZ
Siigg1 = EXCO SMXX Sig1;
* on trace sur la geometrie deformee, c'est plus beau
TRAC CACH Siigg1 Model1 def1;
*
* DEFORMATIONS
*Eps1 = EPSI Model1 Mater1 Depl1;
*TRAC CACH Eps1 Model1;

fin;
\end{Verbatim}
\colorblack

\chapter{Schémas numériques}\label{Ch-shemanum}
\begin{abstract}
Il nous est apparu qu'il était nécessaire d'ajouter quelques mots encore sur les schémas numériques,
notamment après la présentation des chapitres~\ref{Ch-ED} et~\ref{Ch_NewRaph}. Nous resterons
évidemment brefs, tant le sujet est conséquent.
\end{abstract}


\medskip
%\section{Introduction}
Nous nous sommes efforcés de présenter la méthode des éléments finis ainsi que la théorie mathématique la
sous-tendant.

La partie II nous a permis de voir que l'existence de solutions, dans le cas continu, est assurée notamment par
les théorèmes de Lax-Milgram,\index[aut]{Lax (Peter), 1926-, Américain}\index[aut]{Milgram (Arthur Norton), 1912-1961, Américain}
(paragraphe \ref{Sec:LaxMil}) Babuška\index[aut]{Babuška (Ivo Milan), 1926-, Tchèque} (paragraphe \ref{Sec-ThBabuska}) et
Brezzi\index[aut]{Brezzi (Franco), 1945-, Italien}\label{Sec:Brezzi} (paragraphe \ref{Sec-ThBrezzi}) selon le type de formulation.
La partie III, nous a premis de s'assurer de la convergence de la solution approximée vers la solution continue, notamment par les
lemmes de Céa\index[aut]{Cea@Céa (Jean), ?- , Français} (paragraphe \ref{Sec-Cea}) et de
Strang\index[aut]{Strang (William Gilbert), 1934-, Américain} (paragraphe \ref{Sec-Strang}).

Toutefois, nous avons vu lorsque nous prenons en compte certains phénomènes un peu plus complexes (non stationnarité, non linéarité),
nous devons avoir recours à d'autres algorithmes tels que les différences finies, et les algorithmes de Newmark et Newton-Raphson par
exemple. Or nous n'avons rien dit de très général sur la converges des schémas numériques, c'est le pourquoi de ce petit chapitre.


\medskip
\section{Les propriétés d'un schéma numérique}


\medskip
\subsection{Problème bien posé}
Selon une définition due à Hadamard\index[aut]{Hadamard (Jacques Salomon), 1865-1963, Français} à propos des
modèles mathématiques de phénomènes physiques, un \textcolorblue{problème est bien posé} s'il possède les propriétés
suivantes:
\begin{enumerate}
   \item ce problème possède une solution (existence);
   \item cette solution est unique (unicité);
   \item et elle dépend de façon continue des données dans le cadre d'une topologie raisonnable.
\end{enumerate}

\medskip
Typiquement, des problèmes issus de la physique et pour lesquels on peut observer (mesurer) les grandeurs physiques correspondantes
sont bien posés. On pensera notamment au problème de Dirichlet pour l'équation de Laplace et à l'équation de la chaleur (avec conditions
initiales connues).

\medskip
Encore une fois, la mécanique nous montre que tout n'est pas si simple.
Un problème de mécanique des milieux continus est bien posé dans un domaine $\Omega$ si la frontière $\Gamma=\partial\Omega$ de
ce domaine admet une partition en deux sous-ensembles $\Gamma_1$ et $\Gamma_2$ sur lesquels portent des conditions aux limites
sur les déplacements et sur les efforts respectivement (un théorème d'Arnold\index[aut]{Arnold (Douglas Norman), ?-, Américain} existe
dans le cas où les deux conditions existent simultanément, mais nous n'en parlerons pas ici).

\textcolorgreen{Si un problème de mécanique est bien posé, la solution en contrainte existe et est unique, alors que la solution en
déplacement ne l'est pas nécessairement lorsque des mouvements de corps rigides sont possibles.}

\medskip
Les \textcolorblue{problèmes inverses} fournissent souvent des problèmes mal posés.
Par exemple, le problème dit d'i'inversion du temps dans l'équation de la chaleur consistant à déduire une distribution passée de la
température à partir d'un état final n'est pas bien posé. Sa solution est en effet très sensible à des perturbations de l'état final.

\medskip
Comme nous l'avons déjà mentionné, la recherche de la solution d'un problème continu en utilisant des méthodes numériques sur la
discrétisation de ce problème comporte de nombreux types d'erreurs, de la simple erreur d'arrondi dans une donnée au comportement
d'un algorithme. C'est ce dont nous allons parler tout de suite.


\medskip
\subsection{Conditionnement}
Le \textcolorblue{conditionnement} mesure la dépendance de la solution d'un problème numérique par rapport aux données du
problème, ceci afin de contrôler la validité d'une solution calculée par rapport à ces données.
En effet, les données d'un problème numérique dépendent en général de mesures expérimentales et sont donc entachées d'erreurs.

De façon plus générale, on peut dire que le conditionnement associé à un problème est une mesure de la difficulté de calcul numérique
du problème. Un problème possédant un conditionnement bas est dit bien conditionné et un problème possédant un conditionnement élevé
est dit mal conditionné.


\medskip
\subsection{Stabilité ou robustesse}
La \textcolorblue{stabilité} est une propriété de la solution obtenue. Elle se réfère à la propagation des erreurs au cours des
étapes du calcul, à la capacité de l'algorithme de ne pas trop amplifier d'éventuels écarts, à la précision des résultats obtenus,
mais elle ne se limite pas aux erreurs d'arrondis et à leurs conséquences.
Une solution est dite stable si elle est bornée dans l'espace et/ou le temps. La valeur de la stabilité
peut parfois (souvent) être exprimée en fonction du pas de discrétisation.

Les algorithmes dédiés à la résolution d'équations différentielles ou d'équations aux dérivées partielles (en particulier la
méthode des différences finies et la méthode des éléments finis) se basent sur une discrétisation ou un maillage de l'espace
(et du temps): dans ce cas, la stabilité se réfère à un comportement numérique robuste lorsque le pas de discrétisation ou la
taille des mailles tend vers 0.

\textcolorred{La stabilité d'un schéma n'a aucun lien avec la solution exacte du problème traité (convergence).}

\medskip
\colorgris
Une solution $U_n$ est stable si l’on peut trouver une constante $C$ (souvent nulle) telle que:
\begin{equation}
-1 \le \dfrac{U_{n+1}-C}{U_n-C}\le 1, \qquad\forall n
\end{equation}
Ce critère simple permet donc de s'assurer que la solution ne s'éloigne pas d'une valeur de référence.
\colorblack

\medskip
\subsection{Consistance}
La \textcolorblue{consistance} est une propriété de la discrétisation.
Un schéma numérique (d'une équation aux dérivées partielles par exemple) sera dit consistant par rapport au problème
qu'il discrétise si celui-ci tend vers le problème considéré lorsque la discrétisation tend vers 0.
La consistance concerne essentiellement la capacité du schéma à représenter une solution régulière satisfaisant localement les équations aux
dérivées partielles, ceci lorsque les pas de discrétisation ($\Delta t$, $\Delta x$, etc.) tendent tous vers 0.
Plus précisément, si les données d'une étape du traitement algorithmique sont issues d'une solution exacte, les résultats de ce traitement
tendent vers cette solution.

La différence entre l'équation discrétisée et l'équation réelle est appelée l'erreur de troncature

\medskip
\textcolorgris{La consistance d'une discrétisation s'analyse en effectuant un développement en série de Taylor de l'équation
discrétisée et en vérifiant que celle-ci tend vers l'équation originale lorsque le pas de discrétisation tend vers 0.}

\medskip
\subsection{Convergence}
Contrairement à la consistance, qui est une propriété locale, la convergence est de portée globale.

On dit qu'une solution numérique converge vers la solution analytique si elle tend vers elle en tout point
du temps et/ou de l’espace lorsque les paramètres de discrétisation ($\Delta t$, $\Delta x$...) tendent vers 0.

\medskip
C'est évidemment bien la convergence (souvent difficile à prouver) d'un schéma numérique que nous visons, mais
stabilité et consistance (plus faciles à prouver) sont les outils nous permettant de nous en assurer, et ceci via les très beaux
théorèmes de Lax\index[aut]{Lax (Peter), 1926-, Américain}(celui du théorème de Lax-Milgram) et de Lax-Wendroff, que
nous présentons maintenant.\index[aut]{Wendroff (Burton), 1930-, Américain}


\medskip
\section{Théorèmes de convergence}

\begin{theoreme}[Théorème de Lax]\index[aut]{Lax (Peter), 1926-, Américain}\index{théorème! de Lax}
Pour résoudre un problème évolutif avec condition initiale qui est supposé être bien posé, ceci à l'aide d'un schéma numérique
consistant, la stabilité du schéma est une condition nécessaire et suffisante pour assurer sa convergence.
Les notions de consistance, de stabilité et de convergence se réfèrent ici à une même norme.

En d'autres termes: \textcolorred{consistance + stabilité $\Longrightarrow$ stabilité}
\end{theoreme}
Ce théorème est parfois également appelé théorème de Lax–Richtmyer\index[aut]{Richtmyer (Robert Davis), 1910-2003, Américain}

\medskip
Dans les applications physiques que nous considérons, le théorème suivant permet d'assurer la convergence du
schéma numérique utilisé.
\begin{theoreme}[Théorème de Lax-Wendroff]\index[aut]{Wendroff (Burton), 1930-, Américain}\index{théorème! de Lax-Wendroff}
Pour résoudre un problème aux dérivées partielles basé sur une loi de conservation, un schéma numérique qui est à la fois conservatif,
consistant et convergent (lorsque l'on raffine les pas de temps et d'espace, i.e. lorsque $\Delta t \rightarrow 0$ et $\Delta x \rightarrow 0$),
alors la solution numérique converge vers une solution faible des équations.
\end{theoreme}

Notons que Wendroff a passé sa thèse sous la direction de Peter Lax.


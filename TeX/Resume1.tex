\chapter*{Résumé des outils d'analyse fonctionnelle}
%\addcontentsline{toc}{chapter}{Résumé des outils d'analyse fonctionnelle}

\section*{Norme, produit scalaire, espaces}

Soit~$E$ un \textcolorblue{espace vectoriel}.

\medskip
$\|\cdot\|: E\rightarrow \RR_+$ est une \textcolorblue{norme}\index{norme} sur~$E$ ssi elle vérifie:
\begin{enumerate}
  \item~$(\|x\|=0) \Longrightarrow (x=0)$;
  \item~$\forall\lambda\in\RR, \forall x\in E, \|\lambda x\|=|\lambda|\| x\|, \lambda\in\KK$;
  \item~$\forall x,y\in E, \|x+y\|\le\|x\|+\|y\|$ (inégalité triangulaire).
\end{enumerate}

La \textcolorblue{distance issue de la norme} est la distance définie par~$d(x,y)=\|x-y\|$.\index{distance!issue d'une norme}

\medskip
Toute forme bilinéaire symétrique définie positive\index{forme!bilinéaire} 
$\langle.,.\rangle: E\times E\rightarrow\RR$ est un \textcolorblue{produit scalaire} sur~$E$.\index{produit scalaire}
Elle vérifie les propriétés:
\begin{enumerate}
  \item bilinéarité:~$\forall x,y,z\in E, \forall\lambda,\mu\in\RR, \langle x,\lambda y+\mu z\rangle
=\lambda\langle x,y\rangle+\mu\langle x,z\rangle$;
  \item symétrie:~$\forall x,y \in E \quad \langle x\, , \, y \rangle = \langle y\, , \, x\rangle$;
  \item positivité:~$\forall x \in E \quad \langle x\, , \, x \rangle \; \ge \; 0$;
  \item définie:~$\forall x \in E \quad \big(\ \langle x\, , \, x \rangle = 0 \; \Rightarrow x = 0\ \big)$.
\end{enumerate}

La \textcolorblue{norme induite} par le produit scalaire est~$\|x\|=\sqrt{\langle x,x\rangle}$.\index{norme!induite par un produit scalaire}

L'inégalité triangulaire donne alors l'inégalité de Cauchy-Schwarz:~$|\langle x,y\rangle|\le\|x\| \|y\|$.\index{inégalité!de Cauchy-Schwarz}\index[aut]{Cauchy (Augustin Louis, baron -), 1789-1857, Français}\index[aut]{Schwarz (Hermann Amandus), 1843-1921, Allemand}

Un espace vectoriel muni d'une norme est appelé \textcolorblue{espace normé}.\index{espace!vectoriel!normé}

Un espace vectoriel muni d'un produit scalaire est appelé \textcolorblue{espace préhilbertien}, \index{espace!préhilbertien}
qui est donc également un espace normé pour la norme induite.

\medskip
Exemple:
Pour~$E = \RR^n$ et~$x = (x_1, ..., x_n)\in\RR^n$, on a les normes:
\[
\|x\|_1=\dsum_{i=1}^n|x_i| \qquad
\|x\|_2= \sqrt{\dsum_{i=1}^nx_i^2} \qquad
\|x\|_\infty=\sup_i|x_i|
\]

Le produit scalaire défini par~$\langle x,y\rangle=\dsum_{i=1}^nx_iy_i$ a pour norme induite
la norme~$\|\cdot\|_2$.

\medskip
Soit~$E$ un espace vectoriel et~$(x_n)_n$ une suite de~$E$. 
$(x_n)_n$ est une \textcolorblue{suite de Cauchy} ssi 
$\forall\varepsilon>0, \exists N, \forall p> N, \forall q> N: \|x_p-x_q\|>\varepsilon$.

Toute suite convergente est de Cauchy. La réciproque est fausse.

Un espace vectoriel est \textcolorblue{complet} si et seulement si toute suite de Cauchy y est convergente.\index{espace!métrique!complet}

Un \textcolorblue{espace de Banach} est un espace normé complet.\index{espace!de Banach}\index[aut]{Banach (Stephan), 1892-1945, Polonais}

Un \textcolorblue{espace de Hilbert} est un espace préhilbertien complet.\index{espace!de Hilbert}\index[aut]{Hilbert (David), 1862-1943, Allemand}

Un \textcolorblue{espace euclidien}\index[aut]{Euclide, -325-- -265, Grec} est un espace de Hilbert 
de dimension finie.\index{espace!euclidien}






\medskip
\section*{Espaces fonctionnels}

Un \textcolorblue{espace fonctionnel} est un espace vectoriel dont les éléments sont des
fonctions.

Dans la suite, nous considérons les fonctions définies sur un ouvert~$\Omega\subset\RR^n$ et à
valeurs dans~$\RR$ ou~$\RR^p$.

Un fonction~$u$ est \textcolorblue{mesurable} ssi~$\{x/ |u(x)|<r\}$ est mesurable~$\forall r>0$.

On définit les espaces~$\mathscr{L}^p(\Omega)$, pour~$1\le p<\infty$ par:\index{espace!de Lebesgue}\index[aut]{Lebesgue (Henri-Léon), 1875-1941, Français}
\[
\mathscr{L}^p(\Omega)=\left\{u:\Omega\rightarrow\RR, \text{ mesurable, et telle que }
\dint_\Omega|u|^p<\infty\right\}
\]

\medskip
$L^p(\Omega)$ est la classe d'équivalence des fonctions de~$\mathscr{L}^ p(\Omega)$ 
pour la relation d'équivalence <<égalité presque partout>>: on confondra deux fonctions 
dès qu'elles sont égales presque partout, i.e. lorsqu'elles ne différent que sur un ensemble
de mesure nulle.

Les formes:
\[
\|u\|_{L^p}=\left(\dint_\Omega |u|^p\right)^{1/p}, 1\le p<\infty \qquad \text{ et }
\qquad \|u\|_{L^\infty}=\sup_\Omega |u|
\]
ne sont pas des normes sur~$\mathscr{L}^P(\Omega)$
(en effet, $\|u\|_{L^p}=0$ implique que~$u$ est nulle presque partout dans~$\mathscr{L}^p(\Omega)$
et non pas~$u = 0$, d'où la définition des espaces~$L^p(\Omega)$.

Ces formes sont des normes sur~$L^p(\Omega)$ et en font des espaces de Banach (i.e. complets).\index{norme}\index{espace!de Banach}\index[aut]{Banach (Stephan), 1892-1945, Polonais}

\medskip
Dans le cas particulier~$p=2$, on ontient l'espace~$L^2(\Omega)$ des fonctions de carré 
sommable pp. sur~$\Omega$. On remarque que le norme~$\|u\|_{L^2}=\sqrt{\int_\Omega u^2}$
est induite par le produit scalaire~$(u,v)_{L^2}=\int_\Omega uv$, ce qui fait de
l'espace~$L^2(\Omega)$ un espace de Hilbert.\index{produit scalaire!de~$L^2$}





\medskip
\section*{Dérivée généralisée}\index{dérivée!généralisée}

Les éléments des espaces~$L^p$ ne sont pas nécessairement des fonctions très régulières.
Leurs dérivées partielles ne sont donc pas forcément définies partout.
C'est pourquoi on va étendre la notion de dérivation. 
Le véritable outil à introduire pour cela est la notion de distribution.
Une idée simplifiée en est la notion de dérivée généralisée (certes plus limitée que
les distributions, mais permettant de sentir les aspects nécessaires pour aboutir aux formulations
variationnelles).

\medskip
\subsection*{Fonctions tests}\index{fonction test}

On note~$\mathcal{D}(\Omega)$ l'espace des fonctions de~$\Omega$ vers~$\RR$, de
classe~$C^\infty$, et à support compact inclus dans~$\Omega$.
$\mathcal{D}(\Omega)$ est parfois appelé espace des fonctions-tests.

Théorème: \textcolorred{$\overline{\mathcal{D}(\Omega)}=L^2(\Omega)$}

\medskip
\subsection*{Dérivée généralisée}

Soit~$u\in C^1(\Omega)$ et~$v\in\mathcal{D}(\Omega)$. 
Par intégration par parties (ou Green) on a l'égalité:
\[
\dint_\Omega \partial_iu \varphi = -\dint_\Omega u \partial_i\varphi
+\dint_{\partial\Omega} u\varphi n
= -\dint_\Omega u \partial_i\varphi
\]
(car~$\varphi$ est à support compact donc nulle sur~$\partial\Omega$).

Le terme~$\int_\Omega u \partial_i\varphi$ a un sens dès que~$u\in L^2(\Omega)$,
donc~$\int_\Omega \partial_iu \varphi$ aussi, sans que~$u$ ait besoin d'être~$C^1$.
Il est donc possible de définir~$\partial_i u$ même dans ce cas.

\medskip
Cas~$n=1$: Soit~$I$ un intervalle de~$\RR$, pas forcément borné. 
On dit que~$u\in L^2(I)$ admet une dérivée généralisée dans~$L^2(I)$ 
ssi~$\exists u_1\in L^2(I)$ telle que~$\forall\varphi\in\mathcal{D}(I)$,
$\int_I u_1\varphi = \int_I u\varphi'$.

Par itération, on dit que~$u$ admet une \textcolorblue{dérivée généralisée
d'ordre~$k$} dans~$L^2(I)$, notée~$u_k$, ssi:~$\forall\varphi\in\mathcal{D}(I)$,
$\dint_I u_k\varphi = (-1)^k\dint_I u\varphi^{(k)}$.

Ces définitions s'étendent au cas où~$n>1$.

\medskip
Théorème: Quand elle existe, la dérivée généralisée est unique.

Théorème: Quand~$u$ est de classe~$C^1(\overline{\Omega})$ la dérivée généralisée 
est égale à la dérivée classique.




\medskip
\section*{Espaces de Sobolev}

\medskip
\subsection*{Espaces~$H^m$}\index{espace!de Sobolev!$H^m(\Omega)$}

L'\textcolorblue{espace de Sobolev d'ordre~$m$} est défini par:
\[
H^m(\Omega)=\left\{u\in L^2(\Omega)/ \partial^\alpha u\in L^2(\Omega),
\forall\alpha=(\alpha_1, ..., \alpha_n)\in\NN^n \text{ tel que } |\alpha|=\alpha_1+...+\alpha_n\le m
\right\}
\]

On voit que~$H^0(\Omega)=L^2(\Omega)$.

\medskip
$H^m(\Omega)$ est un espace de Hilbert avec le produit scalaire et la norme induite:\index{norme!sur~$H^m(\Omega)$}\index{norme!induite par un produit scalaire}\index{produit scalaire!de~$H^m(\Omega)$}
\[
(u,v)_m=\dsum_{|\alpha|\le m}(\partial^\alpha u,\partial^\alpha v)_0 
\quad \text{ et } \quad
\|u\|_m=\sqrt{(u,u)_m}
\]

Théorème: Si~$\Omega$ est un ouvert de~$\RR^n$ de frontière <<suffisament
régulière>>, alors on a l'inclusion:~$H^m(\Omega)\subset C^k(\overline{\Omega})$
pour~$k<m-\frac{n}2$.

\medskip
\subsection*{Trace}\index{espace!trace}

Dans les problèmes physiques que nous rencontrerons (voir partie 2), nous devrons
pouvoir imposer des conditions aux limites.
Ceci est vrai, que l'on s'intéresse aux formulations forte ou faible.

Pour cela, il faut que la valeur d'une fonction sur la frontière soit définie.
C'est justement ce que l'on apelle sa trace. 
La trace est le prolongement d'une fonction sur le bord de l'ouvert~$\Omega$.

De manière analogue, il est possible de prolonger la définition de la dérivée normale
sur le contour de~$\Omega$, ce qui permet de prendre en compte des conditions
aux limites de type Neumann par exemple.


\medskip
\subsection*{Espace~$H^1_0(\Omega)$}\index{espace!trace}\index{espace!de Sobolev!$H_0^1(\Omega)$}

Soit~$\Omega$ ouvert de~$\RR^n$. 
L'espace~$H^1_0(\Omega)$ est défini comme l'adhérence de~$\mathcal{D}(\Omega)$
pour la norme~$\|\cdot\|_ 1$ de~$H^1(\Omega)$.

\medskip
Théorème: Par construction~$H^1_0(\Omega)$ est un espace complet. 
C'est un espace de Hilbert pour la norme~$\|\cdot\|_ 1$.

\medskip
$H^1_0(\Omega)$ est le sous-espace des fonctions de~$H^1(\Omega)$ de trace nulle 
sur la frontière~$\partial\Omega$ (on a:~$H^1_0(\Omega)=\ker\gamma_0$ où~$\gamma_0$
est l'application trace).

\medskip
Pour toute fonction~$u$ de~$H^1(\Omega)$, on peut définir:
\[
|u|_1=\sqrt{\dsum_{i=1}^n\|\partial_i u\|_ 0^2} =
\sqrt{\dint_\Omega\dsum_{i=1}^n (\partial_i u)^2}
\]

\medskip
Inégalité de Poincaré:\index{inégalité!de Poincaré}\index[aut]{Poincaré (Henri), 1854-1912, Français}
Si~$\Omega$ est borné dans au moins une direction, alors il
existe une constante~$C(\Omega)$ telle que~$\forall u\in H^1_0(\Omega)$; 
$\|u\|_0 \le C(\Omega) |u|_1$.
On en déduit que~$|.|_1$ est une norme sur~$H^1_0(\Omega)$, équivalente à 
la norme~$\|\cdot\|_1$.

\medskip
Corollaire: 
Le résultat précédent s'étend au cas où l'on a une condition de Dirichlet nulle
seulement sur une partie de~$\partial\Omega$, si~$\Omega$ est connexe.

\textcolorgris{On suppose que~$\Omega$ est un ouvert borné connexe, de frontière~$C^1$ 
par morceaux. Soit~$V = \{v\in H^1(\Omega); v = 0 \text{ sur } \Gamma_0\}$
où~$\Gamma_0$ est une partie de~$\partial\Omega$ de mesure non nulle. 
Alors il existe une constante~$C(\Omega)$ telle que~$\forall u\in V$;~$\|u\|_{0,V} \le
C(\Omega) |u|_{1,V}$, où~$\|\cdot\|_{0,V}$ et~$|.|_{1,V}$ sont les norme et semi-norme induites 
sur~$V$.
On en déduit que~$|.|_{1,V}$ est une norme sur~$V$ équivalente à la norme~$\|\cdot\|_{1,V}$.}


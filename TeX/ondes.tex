\chapter{Les ondes}\label{Ch-ondes}
\begin{abstract}
Jusqu'ici, nous n'avons quasiment pas parlé de mode propre cela est plus ou moins évoqué à travers,
par exemple, <<~la plus petite période du système~>> au chapitre précédent... mais cela a été fait
à dessein.

En effet, nous avons voulu, dans ce chapitre, regrouper les approches modales, car cela nous a semblé plus en cohérence.
\end{abstract}

\medskip
\section{Introduction}
Dans la mesure où se document s'adresse à des ingénieurs en mécanique (ou en acoustique),
la notion de mode propre est une notion relativement maîtrisée.

\bigskip
Un système mécanique atteint un \textcolorblue{mode propre de vibration}\index{Mode propre}
lorsque tous les points de ce système sont à une fréquence donnée appelée \textcolorblue{fréquence
propre}\index{Fréquence propre} du système.
Une fréquence propre est \textcolorblue{fondamentale} si elle n'est pas le multiple d'une autre fréquence
propre; dans le cas contraire, c'est une \textcolorblue{harmonique}.

\medskip
On appelle \textcolorblue{résonance}\index{Résonance} le phénomène selon lequel certains
systèmes physiques sont particulièrement sensibles à certaines fréquences.
Un système résonant peut accumuler une énergie, si celle-ci est appliquée sous forme périodique,
et proche d'une fréquence propre.
Soumis à une telle excitation, le système est alors le siège d'oscillations de plus en plus importantes,
jusqu'à atteindre un régime d'équilibre qui dépend des éléments dissipatifs du système, ou bien
jusqu'à rupture d'un composant du système.

\medskip
Si l'on soumet un système résonant à une percussion (pour les systèmes mécaniques) ou à
une impulsion (pour les systèmes électriques), et non plus à une excitation périodique,
alors le système sera le siège d'oscillations amorties, sur des fréquences proches de ses fréquences
propres et retournera progressivement à son état stable.
\textcolorgreen{Physiquement, c'est le coup de marteau de choc donné sur une structure pour en déterminer
les modes.}

\medskip
Un système susceptible d'entrer en résonance, i.e. susceptible d'être le siège d'oscillations
amorties, est un \textcolorblue{oscillateur}\index{Oscillateur}. Un tel système a la particularité de pouvoir
emmagasiner temporairement de l'énergie sous deux formes: potentielle ou cinétique. L'oscillation est le
phénomène par lequel l'énergie du système passe d'une forme à l'autre, de façon périodique.

Si l'on injecte une énergie potentielle au moment où l'énergie potentielle déjà emmagasinée est
maximale, l'énergie ainsi injectée s'ajoute à l'énergie déjà emmagasinée et l'amplitude de l'oscillation
va augmenter, ainsi que l'énergie totale. Idem pour l'énergie cinétique.
Ainsi, si l'on apporte de l'énergie avec une périodicité égale (ou proche) de la périodicité propre du
système, l'énergie totale va augmenter régulièrement et l'amplitude des oscillations du système va
ainsi croître.
L'exemple le plus simple est celui d'une balançoire:
l'énergie de chaque poussée s'ajoute à l'énergie totale, à condition de pousser au bon moment...

Le phénomène de \textcolorblue{résonance}\index{Résonance} n'est rien d'autre que cet effet
d'accumulation de l'énergie en injectant celle-ci au moment où elle peut s'ajouter à l'énergie déjà
accumulée, i.e. <<~en phase~>> avec cette dernière.

\medskip
Quand l'excitation aura cessé, le système résonant sera le siège d'\textcolorblue{oscillations amorties}:
il va revenir plus ou moins vite à son état d'équilibre stable.
En effet, l'énergie de départ sera peu à peu absorbée par les éléments dissipatifs du système
(amortisseur visqueux en mécanique, résistances en électricité...).
Un système peu amorti sera le siège d'un grand nombre d'oscillations qui diminueront lentement avant
de disparaître complètement.

\medskip
La \textcolorblue{représentation modale}\index{Représentation modale} est pertinente dans le domaine
des basses fréquences,  i.e. pour les premiers modes propres.\index{Mode propre}
Dans les domaines moyenne et haute fréquences, on utilise des méthodes adaptées à la
\textcolorblue{densité spectrale}\index{Densité spectrale} élevée.

Les domaines moyenne fréquence et haute fréquence sont définis par la densité spectrale.\index{Densité spectrale}
En effet, l'expression en fréquences n'a pas de sens pour définir ces domaines, une similitude
sur un système physique modifie les fréquences propres\index{Fréquence propre} mais le spectre reste semblable, à
un facteur près.
Dans le cas de fréquences multiples, il existe un sous-espace propre donc les modes propres\index{Mode propre} sont
arbitraires dans ce sous espace. Dans le cas de fréquences voisines (densité spectrale élevée),\index{Densité spectrale}
la représentation modale\index{Représentation modale} n'est pas robuste car de faibles perturbations
du domaine physique vont  entraîner un changement important des modes propres\index{Mode propre}
associés à ces fréquences.
\textcolorblue{Donc la représentation modale n'est pertinente que pour le domaine des basses fréquences,
domaine défini par la densité spectrale.}
Le domaine basse fréquence s'étendra jusqu'à quelques Hz en génie civil, jusqu'à des milliers de Hz
pour de petites structures mécaniques

\medskip
Le phénomène de \textcolorblue{synchronisation},\index{Synchronisation} ou
\textcolorblue{accrochage de fréquences} est un phénomène par lequel deux systèmes excités
chacun selon une fréquences se mettent à osciller selon la même fréquence. On trouve de nombreux
exemples de ce phénomène dans la nature:
\begin{itemize}
   \item Le plus connu et observable est la Lune: la Lune présente toujours la même face à la Terre.
	Cela signifie que la période de rotation de la Lune sur elle-même $T_0$ est égale à la
	période de rotation de la lune autour de la Terre $T\cong 28$~jours. C'est une
	résonance 1:1. L'analyse montre que ce n'est pas une coincidence: cela
	est dû à un faible couplage gravitationnel entre ces deux mouvements.

   \item Un exemple tout aussi connu est celui de la synchronisation des balanciers de deux pendules accrochées
	au même mur d'une pièce. Un faible couplage par les vibrations transmises dans le mur,
	et une légère dissipation, expliquent cet accrochage de fréquences et cette résonance 1:1.

	C'est Huygens\index[aut]{Huygens [ou Huyghens] (Christian), 1629-1695, Néerlandais} qui a
	remarqué et expliqué ce phénomène:
	Le système composé des deux balanciers et du mur a deux fréquences voisines faiblement
	couplées, il possède par le couplage deux modes propres correspondant aux mouvements en
	phase et en opposition de phase des deux pendules. C'est sur le premier mode\index{Mode propre}
	que se produit la synchronisation.\index{Synchronisation}

   \item L'escalier de Cantor,\index[aut]{Cantor (Georg Ferdinand Ludwig Philip), 1845-1918, Allemand}
	ou escalier du diable est un exemple mathématique incontournable en analyse.
	Il correspond au graphe d'une fonction continue $f$, sur $[0,1]$, telle que $f(0)=0$, $f(1)=1$,
	qui est dérivable presque partout, la dérivée étant presque partout nulle.

	L'escalier de Cantor peut également être vu comme la fonction de répartition d'une variable
	aléatoire réelle continue qui n'est pas à densité, et qui est même étrangère à la
	mesure de Lebesgue.\index[aut]{Lebesgue (Henri-Léon), 1875-1941, Français}

	Enfin, l'escalier du diable peut aussi être vu comme résultant d'un phénomène de
	synchronisation.\index{Synchronisation}
	Si on change un paramètre extérieur du système de façon lente et continue, par
	exemple l'amplitude $\alpha_0\in\RR$, alors la valeur de l'accrochage $a=p/q$ va dépendre de
	ce paramètre. On obtient alors une fonction $a(\alpha_0)$ de $\RR$ dans les rationnels $\QQ$.
	Cette fonction très étrange comporte une multitude de paliers plus ou moins larges... et correspond
	à l'escalier du diable.
	
   \item On retrouve la synchronisation dans de nombreux phénomènes naturels : vols d'oiseau,
	clignotement des lucioles...
\end{itemize}
%%%%%%%%%%%%%%%%%%
\medskipvm
En mathématiques, le concept de vecteur propre\index{Vecteur propre} est une notion portant sur une
application linéaire d'un espace dans lui-même. Il correspond à l'étude des axes privilégiés, selon
lesquels l'application se comporte comme une dilatation, multipliant les vecteurs par une même constante.
Ce rapport de dilatation est appelé valeur propre;\index{Valeur propre} les vecteurs auxquels il s'applique s'appellent
vecteurs propres, réunis en un espace propre.\index{Espace!propre}

\medskip
\begin{histoire}%
Bien qu'existant sous une forme non formalisée depuis longtemps, il aura fallu attendre l'invention
des structures algébriques nécessaires pour vraiment pouvoir parler des valeurs propres\index{Valeur propre}
(issues par exemple de la cloture algébrique de $\CC$ démontrée par Gauß).\index[aut]{Gau\ss{} (Johann Carl Friedrich), 1777-1855, Allemand}

\medskip
L'exemple immédiat qui vient à l'esprit est le traitement de l'équation de la chaleur\index{ED-EDP!de la chaleur} par
Fourier\index[aut]{Fourier (Jean Baptiste Joseph), 1768-1830, Français} qui utilise déjà
une base de vecteurs propres,\index{Vecteur propre} bien que le concept n'ait pas encore été
défini. Hamilton\index[aut]{Hamilton (William Rowan, Sir -), 1805-1865, Irlandais} introduira
la notion de polynôme caractéristique, ce qui permet de déterminer ce
que l'on appelle maintenant les valeurs propres\index{Valeur propre} associées à l'endomorphisme d'une ED linéaire.

\sbox{\MaBoiteAvecPhotos}{\setlength{\tabcolsep}{0pt}\scriptsize%
\begin{tabular}{cccc}
\includegraphics[height=\the\HauteurDesPhotos]{Cayley}&
\includegraphics[height=\the\HauteurDesPhotos]{Grassmann}&
\includegraphics[height=\the\HauteurDesPhotos]{Sylvester}&
\includegraphics[height=\the\HauteurDesPhotos]{Jordan}\\
Cayley&Grassmann&Sylvester&Jordan%
\end{tabular}}
\medskip
\ImageADroite{%
Plusieurs aller-retour permettront de définir les notions d'espace vectoriel (Cayley,\index[aut]{Cayley (Arthur), 1821-1895, Anglais}
Grassmann,\index[aut]{Grassmann (Hermann Günther), 1809-1877, Allemand}
Cauchy),\index[aut]{Cauchy (Augustin Louis, baron -), 1789-1857, Français}
de matrice (Sylvester,\index[aut]{Sylvester (James Joseph), 1814-1897, Anglais}
Cayley)\index[aut]{Cayley (Arthur), 1821-1895, Anglais} et de valeurs propres\index{Valeur propre}
(Sylvester,\index[aut]{Sylvester (James Joseph), 1814-1897, Anglais}
Jordan).\index[aut]{Jordan (Marie Ennemond Camille), 1838-1922, Français}
Hilbert\index[aut]{Hilbert (David), 1862-1943, Allemand} finalement fera prendre
conscience de la profondeur de la notion de valeur propre.\index{Valeur propre}
L'analyse fonctionnelle naît dans la foulée, et elle est l'objet de la première partie de ce document.
}
\end{histoire}
\colorblack

\medskip
\section{Notions de valeur, vecteur, mode et fréquence propres}\index{Valeur propre}\index{Vecteur propre}\index{Mode propre}\index{Fréquence propre}

\medskip
Étant donné une matrice carrée $\MM{A}$ d'ordre $n$ \textcolorgris{(à coefficients dans un anneau commutatif)},
on cherche un polynôme dont les racines sont précisément les valeurs propres de $\MM{A}$.
Ce polynôme est appelé \textcolorblue{polynôme caractéristique}\index{Polynôme! caractéristique}
de $\MM{A}$ et est défini par:
\begin{equation} p_A(X):=\det(X\MM{I}_n-\MM{A})\end{equation}
avec $X$ l'indéterminée du polynôme et $\MM{I}_n$ la matrice identité d'ordre $n$.
%%%%%%%%%%%%%%%%%%%%%%
\medskipvm
Si $\lambda$ est une \textcolorblue{valeur propre}\index{Valeur propre} de $\MM{A}$, alors il existe un
\textcolorblue{vecteur propre}\index{Vecteur propre} $\VV{V}$ non nul tel que $\MM{A}\VV{V} = \lambda \VV{V}$, i.e.
tel que l'on ait $(\lambda \MM{I}_n-\MM{A})\VV{V} = \VV{0}$.
%%%%%%%%%%%%%%%%%%%%%%
\medskipvm
Puisque $\VV{V}$ est non nul, cela implique que la matrice $\lambda \MM{I}_n-\MM{A}$ est singulière, donc de déterminant nul.
%%%%%%%%%%%%%%%%%%%%%%
\medskipvm
Cela montre que les valeurs propres de $\MM{A}$ sont des zéros de la fonction $\lambda\mapsto \det(\lambda \MM{I}_n - \MM{A})$
i.e. des racines du polynôme $\det(X\MM{I}_n-\MM{A})$.
%%%%%%%%%%%%%%%%%%%%%%
\medskipvm
La propriété la plus importante des polynômes caractéristiques\index{Polynôme! caractéristique} est que
les valeurs propres\index{Valeur propre} de $\MM{A}$ sont exactement les racines du polynôme $p_A(X)$.

\medskip
Quelques propriétés importantes:
\begin{itemize}
   \item $p_A(X)$ est un polynôme unitaire (coefficient dominant égal à 1) et son degré
	est égale à $n$.
   \item $\MM{A}$ et sa transposée ont le même polynôme caractéristique.
   \item Deux matrices semblables ont le même polynôme caractéristique. ($\MM{A}$ et $\MM{B}$ sont semblables s'il
	existe une matrice inversible $\MM{P}$ telle que $\MM{A} = \MM{P}\MM{B}\MMI{P}$).
	Attention, la réciproque n'est pas vraie en général.
   \item Si $p_A(X)$ peut être décomposé en produit de facteurs de degré 1, alors $\MM{A}$ est semblable
	à une matrice triangulaire (et même à une matrice de Jordan).
\end{itemize}

\medskip
\colorgris
\paragraph{Pour aller un peu plus loin}
Le \textcolorblue{théorème de Cayley-Hamilton}\index[aut]{Cayley (Arthur), 1821-1895, Anglais}\index[aut]{Hamilton (William Rowan, Sir -), 1805-1865, Irlandais}\index{Théorème!de Cayley-Hamilton}
(dont la première démonstration est due à Frobenius)\index[aut]{Frobenius (Ferdinand Georg), 1849-1917, Allemand}
affirme que tout endomorphisme d'un espace vectoriel de dimension finie $n$ sur un corps commutatif quelconque
annule son propre polynôme caractéristique.

En termes de matrice, cela signifie que : si $\MM{A}$ est une matrice carrée d'ordre $n$ et si
$p_A(X)$ est son polynôme caractéristique, alors en remplaçant formellement
$X$ par la matrice $\MM{A}$ dans le polynôme, le résultat est la matrice nulle, i.e.:
\begin{equation}p_A(\MM{A})= \MM{A}^n + p_{n-1}\MM{A}^{n-1} + \ldots + p_1 \MM{A} + p_0 \MM{I}_n = \MM{0}_n \end{equation}

Cela signifie que le polynôme caractéristique\index{Polynôme! caractéristique} est un
polynôme annulateur de $\MM{A}$.
Les applications sont importantes car le polynôme minimal (qui est l'unique polynôme unitaire qui
engendre l'idéal annulateur de l'ensemble des polynômes qui annulent l'endomorphisme dont $\MM{A}$
est la représentation) cache une décomposition en somme  directe de sous-espaces stables.
\colorblack











\medskip
\section{Vibration des structures}
Revenons sur l'équation de la dynamique sous forme matricielle:
\begin{equation} \MM{M}\VV{\ddot{q}}+\MM{C}\VV{\dot{q}}+\MM{K}\VV{q}=\VV{F} \end{equation}
qui peut être vue comme l'équation de la statique $\MM{K}\VV{q}=\VV{F}$ à laquelle on
ajoute des forces \emph{extérieures} d'inertie $-\MM{M}\VV{\ddot{q}}$ et des forces \emph{extérieures} visqueuses $-\MM{C}\VV{\dot{q}}$.
%%%%%%%%%%%%%%%%%%%%%%
\medskipvm
D'un point de vue pratique, on distingue 3 types de problèmes:
\begin{itemize}
   \item détermination d'une \textcolorblue{réponse libre}:\index{Vibration! libre}
	dans ce cas, la sollicitation est nulle $\VV{F} = \VV{0}$;
   \item détermination d'une \textcolorblue{réponse périodique}:\index{Vibration! périodique}
 	dans ce cas, la sollicitation $\VV{F}$ est périodique;
   \item détermination d'une \textcolorblue{réponse transitoire}:\index{Vibration! transitoire}
	dans ce cas, la sollicitation $\VV{F}$ est quelconque.
\end{itemize}
Dans les deux premiers cas, les conditions initiales du système n'ont \textcolorblue{aucune importance}.
On cherche à déterminer une solution générale.
%%%%%%%%%%%%%%%%%%%%%%
\medskipvm
On pourrait considérer que le chapitre précédent \ref{Ch-temps} répond à ces trois
types de problèmes, ce qui n'est pas fondamentalement faux.
Toutefois, il est plus judicieux de considérer que seul le cas de la  \textcolorblue{dynamique transitoire}
y a été explicité, et encore uniquement pour l'aspect non modal (qui sera développé dans ce chapitre
un peu plus loin).

En effet, dans les cas des  \textcolorblue{vibrations libres (amorties ou non) ou périodique forcées},
il est possible d'utiliser la notion de mode, qui n'avait pas été abordée dans les chapitres précédents.


\medskip
\subsection{Vibrations libres non amorties}\index{Vibration!libre}
En l'absence de sollicitation et d'amortissement, l'équation de la dynamique devient:\index{ED-EDP!relation fondamentale de la dynamique}
\begin{equation} \MM{M}\VV{\ddot{q}}+\MM{K}\VV{q}=\VV{0} \end{equation}
dont la solution générale est harmonique et s'écrit:
\begin{equation} \VV{q}=\VV{\overline{q}}\mathrm{e}^{i\omega t} \end{equation}
En injectant la forme de la solution générale dans l'équation de la dynamique, on voit que la
pulsation $\omega$ est solution du problème de valeurs propres suivant:
\begin{equation} \MM{K}\VV{\overline{q}}=\omega^2\MM{M}\VV{\overline{q}} \end{equation}
ce qui conduit à:
\begin{equation} \det\left(\MM{K}-\omega^2\MM{M}\right)=0\end{equation}
%%%%%%%%%%%%%%%%%%%%%%%%
\medskipvm
On obtient ainsi les \textcolorblue{$n$ valeurs propres}\index{Valeur propre}
$\omega_1,\ldots,\omega_n$, où $n$ est la taille du système (i.e. les matrices $\MM{M}$ et $\MM{K}$ sont $n\times n$).
%%%%%%%%%%%%%%%%%%%%%%%%
\medskipvm
On trouve également les $n$ vecteurs $\VV{\overline{q}}_i$ appelés \textcolorblue{modes propres
du système}\index{Mode propre} et que l'on \textcolorred{norme par rapport à la masse}, i.e. tels que:
\begin{equation} \LL{\overline{q}}_i\MM{M}\VV{\overline{q}}_i=1 \qquad  (\forall i\in[1,n]) \end{equation}
%%%%%%%%%%%%%%%%%%%%%%
\medskipvm
\textcolorgreen{La détermination des valeurs propres $\omega_i$ se fait rarement en cherchant
les zéros de l'équation du déterminant en raison de la très grande taille du système dans le cas
général, et des considérables différences d'ordre de grandeur entre les valeurs propres.}

De toutes façons, ce sont les premières fréquences qui déterminent le comportement du système.

\medskip
\colormagenta
\paragraph{Rappel du cas unidimensionnel}
L'équation différentielle à résoudre est $M\ddot{u}+Ku=0$,
dont la solution s'écrit: \begin{equation} u=A\sin (\omega_0 t)+B\cos(\omega_0 t) \end{equation}
La \textcolorblue{pulsation propre} du système $\omega_0$, sa \textcolorblue{fréquence propre} $f_0$ et sa \textcolorblue{période propre} $T_0$ sont définies et reliées par les relations:
\begin{equation}
\omega_0=\sqrt{\dfrac{K}M} \qquad\qquad f_0=\dfrac{\omega_0}{2\pi} \qquad\qquad T_0=\dfrac1{f_0}
\end{equation}
Les constantes d'intégration sont déterminées à l'aide des CL sur le déplacement $u_0$ et la
vitesse $\dot{u}_0$. Il vient: \begin{equation} A=\frac{\dot{u}_0}{\omega_0} \quad\text{ et }\quad B=u_0\end{equation}

\medskip\colorblack
Des méthodes permettant de trouver les premiers zéros d'un polynôme de degré $n$ ont donc été
mises au point.
La majeure partie de ces méthodes consiste à écrire la relation du déterminant sous la forme suivante:
\begin{equation} \MM{H}\VV{X}=\lambda\VV{X} \end{equation}
où $\MM{H}$ est une matrice définie positive.
%%%%%%%%%%%%%%%%%%%%%%%%
\medskipvm
On se sert pour cela de la décomposition de Cholesky\index[aut]{Cholesky (André-Louis), 1875-1918, Français}
de $\MM{K}$, i.e. en l'écrivant à partir d'une matrice triangulaire inférieure $\MM{L}$ sous la forme $\MM{K}=\MMT{L}\MM{L}$.
%%%%%%%%%%%%%%%%%%%%%%%%
\medskipvm
L'équation du déterminant conduit alors à:
\begin{equation} \MMI{L}\MM{M}\MMIT{L}\MMT{L}\VV{\overline{q}}=\frac1{\omega^2}\MMT{L}\VV{\overline{q}} \end{equation}
et l'on obtient la forme cherchée en posant $\lambda=\omega^{-2}$, $\VV{X}=\MMT{L}\VV{\overline{q}}$
et $\MM{H}=\MMI{L}\MM{M}\MMIT{L}$, où $\MM{H}$ est bien symétrique.

\medskip
\colorgris
Si $\MM{M}$ et $\MM{K}$ sont définies positives (ce qui est le cas habituel des problèmes en dynamique
des structures), il existe $n$ valeurs propres réelles positives. Ces solutions sont appelées
pulsations propres du système.

Si $\MM{K}$ est singulière (elle ne possède pas d'inverse), alors, afin de pouvoir utiliser les méthodes
précédentes, on utilise un artifice qui consiste à introduire un paramètre $\alpha\in\RR$ du
même ordre de grandeur que $\omega^2$. On doit alors résoudre:
\begin{equation}\left(\MM{K}+\alpha \MM{M}\right)\VV{\overline{q}} = (\omega^2+\alpha)\MM{M} \VV{\overline{q}}\end{equation}
La nouvelle matrice $\MM{K}+\alpha \MM{M}$ est alors inversible et la solution cherchée est $\omega^2+\alpha$.
\colorblack

\medskip
\subsection{Vibrations libres amorties}\index{Vibration!libre}
\subsubsection{Problèmes du premier ordre}
Si $\MM{M}=\MM{0}$, l'équation de la dynamique\index{ED-EDP!relation fondamentale de la dynamique} se
transforme en celle de la chaleur:\index{ED-EDP!de la chaleur}
\begin{equation} \MM{C}\VV{\dot{q}}+\MM{K}\VV{q}=\VV{0} \end{equation}
dont on cherche une solution générale sous la forme:
\begin{equation} \VV{q}=\VV{\overline{q}}\mathrm{e}^{-\omega t} \end{equation}
ce qui conduit au problème de valeurs propres:\index{Valeur propre}
\begin{equation} \left(\MM{K}-\omega\MM{C}\right)\VV{\overline{q}}=0\end{equation}

Les matrices $\MM{C}$ et $\MM{K}$ sont généralement définies positives, donc $\omega$ est réelle positive. La solution présente un terme de décroissance exponentielle qui ne correspond pas réellement à un état de régime permanent.

\medskip
\subsubsection{Problèmes du second ordre}
Dans le cas général ($\MM{M}\ne\MM{0}$), on doit donc résoudre l'équation de la dynamique\index{ED-EDP!relation fondamentale de la dynamique}
sans sollicitation:
\begin{equation} \MM{M}\VV{\ddot{q}}+\MM{C}\VV{\dot{q}}+\MM{K}\VV{q}=\VV{0} \end{equation}
dont on cherche une solution générale sous la forme:
\begin{equation} \VV{q}=\VV{\overline{q}}\mathrm{e}^{-\alpha t} \end{equation}
avec $\alpha\in\CC$.
Cela conduit au problème de valeurs propres:\index{Valeur propre}
\begin{equation} \left(\alpha^2\MM{M}+\alpha\MM{C}+\MM{K}\right)\VV{\overline{q}}=0\end{equation}
où $\VV{\overline{q}}\in\CC$.

La partie réelle de la solution représente une vibration amortie.
Ce problème est plus délicat à résoudre que les précédents si bien que la résolution
explicite est peu courante.

\medskip
\colormagenta
\paragraph{Rappel du cas unidimensionnel}
L'équation différentielle à résoudre est: $M\ddot{u}+C\dot{u}+Ku=0$, que l'on
réécrit, en introduisant le coefficient d'amortissement $\xi$:
\begin{equation}\ddot{u}+2\xi\omega_0\dot{u}+\omega^2_0 u=0\end{equation}

La solution s'écrit: \begin{equation} u=\left[A\sin (\omega_D t)+B\cos(\omega_D t)\right] \mathrm{e}^{-\xi\omega_0t} \end{equation}
où $\omega_D$, la \textcolorblue{pseudo-pulsation}, est definie par:
\begin{equation}\omega_D=\omega_0\sqrt{1-\xi^2}\end{equation}
On remarquera que $\omega_D$ n'est défini que pour $\xi<e1$, i.e.
dans le cas des systèmes \textcolorblue{sous-critiques} ou \textcolorblue{sous-amortis}.

Les constantes d'intégration sont déterminées à l'aide des conditions aux limites sur le déplacement $u_0$ et la
vitesse $\dot{u}_0$. Il vient: \begin{equation} A=\dfrac{\dot{u}_0+u_0\xi\omega_0}{\omega_0} \quad \text{ et } B=u_0\end{equation}

Le système amorti oscille à une pulsation $\omega_D$ (légèrement) inférieure à la pulsation du
système non amorti $\omega_0$. Si l'amortissement est positif (ce qui n’est parfois pas le cas
pour des systèmes instables), l'amplitude du mouvement décroît dans le temps de façon
exponentielle (en atteignant une amplitude nulle mais pour un temps infini).

\medskip
Dans le cas d'un \textcolorblue{système sur-amorti} ($\xi>1$), alors, en posant $\omega'_D=\sqrt{\xi^2-1}$, la solution est du type:
\begin{equation} u=\left[A\mathrm{e}^{\omega'_D t}+B\mathrm{e}^{-\omega'_D t}\right] \mathrm{e}^{-\xi\omega_0t} \end{equation}
avec les constantes d'intégration:
\begin{equation}A=\dfrac{\dot{u}_0+(\omega'_D+\xi\omega_0)u_0}{2\omega'_D} \quad \text{ et }\quad
B=\dfrac{-\dot{u}_0+(\omega'_D-\xi\omega_0)u_0}{2\omega'_D} \end{equation}
Un système sur-amorti n'est pas un oscillateur...

\medskip
Dans le cas d'un \textcolorblue{système critique} ($\xi=1$), alors la solution s'écrit:
\begin{equation}
u = (u_0+\omega_0u_0t+\dot{u}_0t)\mathrm{e}^{-\omega_0 t}
\end{equation}
Ce système n'est pas lui non plus un oscillateur...




\medskip\colorblack
\subsection{Vibrations périodiques forcées}\index{Vibration! périodique}
Il s'agit du cas où la sollicitation est périodique. Nous l'écrirons sous la forme:
\begin{equation} \VV{F}=\VV{\overline{F}}\mathrm{e}^{\alpha t} \end{equation}
avec $\alpha=\alpha_1+i\alpha_2\in\CC$.

La solution générale s'écrit:
\begin{equation} \VV{q}=\VV{\overline{q}}\mathrm{e}^{\alpha t} \end{equation}
En substituant cette forme de solution dans l'équation de la dynamique, il vient:\index{ED-EDP!relation fondamentale de la dynamique}
\begin{equation} \left(\alpha^2\MM{M}+\alpha\MM{C}+\MM{K}\right)\VV{\overline{q}}=\MM{D}\VV{\overline{q}}=-\VV{\overline{F}} \end{equation}
qui \textcolorred{n'est pas un problème de valeurs propres}, mais ce système peut être résolu
comme un problème statique, i.e. en inversant la matrice $\MM{D}$. Attention, la solution appartient à $\CC$.

\medskip
On sépare alors les parties réelle et imaginaire en notant:
$\mathrm{e}^{\alpha t}=\mathrm{e}^{\alpha_1 t}(\cos(\alpha_2 t)+i\sin(\alpha_2 t)$, $\VV{\overline{F}}=\VV{\overline{F}}_1
+i\VV{\overline{F}}_2$ et $\VV{\overline{q}}=\VV{\overline{q}}_1+i\VV{\overline{q}}_2$.

On obtient alors le système suivant:
\begin{equation}
\MM*{(\alpha_1^2-\alpha_2^2)\MM{M}+\alpha_1\MM{C}+\MM{K} & -2\alpha_1\alpha_2\MM{M}-\alpha_2\MM{C}\\
2\alpha_1\alpha_2\MM{M}+\alpha_2\MM{C} & (\alpha_1^2-\alpha_2^2)\MM{M}+\alpha_1\MM{C}+\MM{K}}
\VV*{\VV{\overline{q}}_1\\ \VV{\overline{q}}_2}
=
-\VV*{\VV{\overline{F}}_1\\ \VV{\overline{F}}_2}
\end{equation}
dans lequel toutes les quantités sont réelles.

Il est ainsi possible de déterminer la réponse à toute excitation périodique par résolution directe.

\medskip
\textcolorgreen{Pour une excitation périodique, la réponse après une phase transitoire initiale
n'est plus influencée par les conditions initiales.
La solution obtenue représente la réponse qui s'établit.
Ceci est valable aussi bien pour les problèmes en dynamique des structures que pour les problèmes
de conduction de chaleur.}

\medskip
\colormagenta
\paragraph{Rappel du cas unidimensionnel}
L'équation différentielle à résoudre est: $M\ddot{u}+C\dot{u}+Ku=F(t)$, que l'on
réécrit: \begin{equation} \ddot{u}+2\xi\omega_0\dot{u}+\omega^2_0 u=F_0/M \cos(\omega t)\end{equation}
en supposant que $F(t)$ est un chargement monofréquentiel.

La solution est somme d'une solution particulière, appelée régime permanent ou forcé, et d'une
combinaison linéaire de l'équation sans second membre, dit régime transitoire.

On voit alors, de manière intuitive, que:
\begin{itemize}%\colorgreen
   \item La fréquence du régime permanent (ou forcé) est celle de la fréquence d'excitation (qui <<~force~>> le système);
   \item La fréquence du régime transitoire est la fréquence propre du système (puisqu'il n'y a pas de second membre).
\end{itemize}
La solution s'écrit:
\begin{equation}
u = \dfrac{F_0}M\dfrac{\cos(\omega t-\theta)}{\sqrt{\left(1-\frac{\omega^2}{\omega_0^2}\right)^2+\left(2\xi\frac{\omega}{\omega_0}\right)^2}}
+\left[A\sin (\omega_D t)+B\cos(\omega_D t)\right] \mathrm{e}^{-\xi\omega_0 t}
\end{equation}

\medskip
En présence d'amortissement, le régime transitoire disparaît après quelques prériodes d'oscillation.






\medskip\colorblack
\subsection{Régimes transitoires}\label{Sec-RT}\index{Vibration! transitoire}
Le lecteur aura sans doute remarqué que dans les méthodes présentées ci-dessus,
les conditions initiales du problème ne sont pas prises en compte
Par exemple le comportement sismique des structures ou l'évolution transitoire d'un problème
de conduction de chaleur nécessitent de prendre en compte à la fois les conditions
initiales et le caractère non périodique des sollicitations.

L'obtention d'une solution à ce genre de problème nécessite soit l'utilisation d'une discrétisation
dans le domaine temporel (voir chapitre \ref{Ch-temps}, soit l'utilisation de méthodes adaptées.
Dans ce dernier cas, il existe deux approches:
\begin{itemize}
   \item la méthode de réponse en fréquence;
   \item la méthode d'analyse modale.
\end{itemize}
Nous allons maintenant présenter cette dernière méthode.

\medskip
\subsubsection{Décomposition modale}\index{Décomposition modale}
La méthode de décomposition modale est sans doute l'une des plus importantes et des plus
employées.
%%%%%%%%%%%%%%%%%%
\medskipvm
Nous partons toujours de notre équation de la dynamique sous la forme:\index{ED-EDP!relation fondamentale de la dynamique}
\begin{equation} \MM{M}\VV{\ddot{q}}+\MM{C}\VV{\dot{q}}+\MM{K}\VV{q}+\VV{F} = \VV{0} \end{equation}
%%%%%%%%%%%%%%%%%%
\medskipvm
Nous avons vu qu'en réponse libre ($\VV{F}=\{0\}$), la solution s'écrit:
\begin{equation} \VV{q} = \VV{\overline{q}} \mathrm{e}^{-\alpha t} = \dsum_{i=1}^n \VV{\overline{q}}_i \mathrm{e}^{-\alpha_i t} \end{equation}
où $\alpha_i$ sont les valeurs propres et $\VV{\overline{q}}_i$ les vecteurs propres.
%%%%%%%%%%%%%%%%%%
\medskipvm
Pour la réponse forcée, l'idée consiste à \textcolorblue{chercher la solution sous la forme
d'une combinaison linéaire des modes propres}, i.e. sous la forme:
\begin{equation} \VV{q} = \dsum_{i=1}^n \VV{\overline{q}}_i y_i(t) \end{equation}
où la quantité $y_i(t)$ représente la contribution de chaque mode.
%%%%%%%%%%%%%%%%%%
\medskipvm
En injectant cette forme de solution dans l'équation de la dynamique (puis en composant à gauche par
$\LL{\overline{q}}_i$), on obtient un \textcolorblue{ensemble d'équations scalaires indépendantes}:
\begin{equation} m_i\ddot{y}_i+c_i\dot{u}_i+k_iy_i+F_i = 0 \end{equation}
dont les paramètres sont, grâce à l'orthogonalité des modes:\index{Propriété d'orthogonalité}
\begin{equation}\left\{
\begin{aligned}
	m_i &= \LL{\overline{q}}_i\MM{M}\VV{\overline{q}}_i\\
	c_i &= \LL{\overline{q}}_i\MM{C}\VV{\overline{q}}_i\\
	k_i &= \LL{\overline{q}}_i\MM{K}\VV{\overline{q}}_i\\
	F_i &= \LL{\overline{q}}_i\VV{F}
\end{aligned}
\right.\end{equation}
Les équations scalaires se résolvent ensuite par des méthodes élémentaires indépendamment
les unes des autres. Le vecteur total est ensuite obtenu par superposition.
%%%%%%%%%%%%%%%%%%
\medskipvm
Toutefois, pour effectuer cette superposition, il n'aura pas échappé au lecteur qu'il faut
avoir résolu le problèmes des valeurs propres.\index{Valeur propre}
Dans le cas général, le calcul des valeurs et des vecteurs propres \textcolorred{complexes} est loin
d'être facile.
La méthode habituelle consiste à déterminer les valeurs propres \textcolorblue{réelles} du
problème vu précédemment:
\begin{equation} \omega^2\MM{M}\VV{\overline{q}}=\MM{K}\VV{\overline{q}} \end{equation}
On montre que le problème est découplé en $y$ seulement si on a la \textcolorblue{propriété
d'orthogonalité de $\MM{C}$}: \begin{equation}\LL{\overline{q}}_i\MM{C}\VV{\overline{q}}_i=0\end{equation}
%%%%%%%%%%%%%%%%%%
\medskipvm
\textcolorred{Or ceci n'est pas vrai en général} car les vecteurs propres assurent uniquement
l'orthogonalité de $\MM{M}$ et $\MM{K}$.\index{Propriété d'orthogonalité}
%%%%%%%%%%%%%%%%%%
\medskipvm
En revanche, si la matrice d'amortissement $\MM{C}$ est une combinaison linéaire des matrices $\MM{M}$ et
$\MM{K}$, la propriété d'orthogonalité est alors évidemment satisfaite.\index{Propriété d'orthogonalité}
C'est l'\textcolorblue{hypothèse de Basile}.

\textcolorblue{Dans la suite on suppose que la propriété d'orthogonalité de $\MM{C}$ est satisfaite}.\index{Propriété d'orthogonalité}
L'équation sur $\omega$ devient alors:
\begin{equation} \omega_i^2\MM{M}\VV{\overline{q}}_i =\MM{K}\VV{\overline{q}}_i \end{equation}
et par suite:
\begin{equation} \omega_i^2m_i = k_i\end{equation}
%%%%%%%%%%%%%%%%%%%%%
\medskipvm
\textcolorblue{En supposant que les modes sont normalisés} de telle sorte que $m_i=1$ et en
posant $c_i = 2\omega_i^2c'_i$ (où \textcolorgreen{$c'_i$ correspond au pourcentage d'amortissement
par rapport à sa valeur critique}), on montre que les équations scalaires se mettent sous forme d'une ED
du second ordre :
\begin{equation} \ddot{y_i}+2\omega_i^2c'_i \dot{y_i} + \omega_i^2y_i + F_i = 0 \end{equation}
dont la solution générale est:
\begin{equation} y_i = \dint_0^t F_i \mathrm{e}^{-c'_i\omega_i(t-\tau)}\sin \omega_i (t-\tau) \dd\tau \end{equation}
%%%%%%%%%%%%%%%%%%%%%
\medskipvm
Une intégration numérique permet de déterminer une réponse, puis la superposition de ces termes
donne la réponse transitoire totale (en principe !).
%%%%%%%%%%%%%%%%%%
\medskipvm
\textcolorgreen{On rappelle que la méthode de décomposition modale\index{Décomposition modale}
nécessiterait la détermination  de l'ensemble des valeurs et modes propres,\index{Valeur propre}\index{Mode propre}
ce qui représenterait des calculs considérables.
D'un point de vue pratique, on ne prend en compte qu'un nombre limité de modes étant donné que
les réponses à des fréquences élevées sont souvent très amorties et prennent par conséquent
des valeurs négligeables. Par ailleurs le problème des hautes fréquences n'est souvent abordé
que de manière statistique.}

\medskip
\subsection{Calcul des modes propres et méthodes de réduction modale}
Les \textcolorblue{méthodes de réduction modale}, ont pour but d'effectuer un
changement de base dans l'étude d'une structure: on souhaite remplacer l'espace vectoriel initial, dont la dimension
est égale au nombre de ddl, par un autre, dont la taille sera inférieure.
\textcolorgreen{En d'autres termes, on cherche une base plus optimale afin de diminuer la taille de cet espace vectoriel, tout en s'assurant
que ce qui n'est pas pris en compte est bien négligeable.
Or, physiquement, on s'aperçoit que les modes propres (et surtout les premiers modes propres) réalisent cet optimal.}

\medskip
Il existe deux principaux types de méthodes:
\begin{itemize}
   \item les méthodes à interfaces libres (Craig...);\index[aut]{Craig (Roy R. Jr), ?, Américain}
   \item et les méthodes à interfaces fixes (Craig-Bampton).\index[aut]{Bampton (Mervyn Cyril Charles), ?, Américain}\index[aut]{Craig (Roy R. Jr), ?, Américain}
\end{itemize}
Dans ce document, nous ne présenterons que cette dernière, qui, de plus,
est particulièrement adaptée au cas de sous-structuration, i.e. lorsque le système considéré est scindé en
sous-structures.
%%%%%%%%%%%%%%%%
\medskipvm
Mais tout d'abord, commençons par présenter succinctement quelques méthodes de calcul des modes propres, ce qui n'est
pas si aisé que cela, et peut  s'avérer coûteux selon les méthodes et le nombre de modes calculés.

\medskip
\subsubsection{Quotient de Rayleigh}\index[aut]{Rayleigh (John William Strutt, troisième baron -), 1842-1919, Anglais}\index{Quotient de Rayleigh}

\begin{definition}[Matrice hermitienne]
Une matrice hermitienne, ou auto-adjointe, est une matrice $\MM{A}$ carrée à éléments complexes  telle que
cette matrice est égale à la transposée de sa conjuguée, i.e.:
\begin{equation}
\MM{A} = \overline{\MM{A}}^T
\end{equation}
\end{definition}
En particulier, une matrice à éléments réels est hermitienne si et seulement si elle est symétrique.
%%%%%%%%%%%%%%%%%%
\medskipvm
Une matrice hermitienne est orthogonalement diagonalisable et toutes ses valeurs propres sont
réelles. Ses sous-espaces propres sont deux à deux orthogonaux.

\begin{definition}[Quotient de Rayleigh]\index[aut]{Rayleigh (John William Strutt, troisième baron -), 1842-1919, Anglais}\index{Quotient de Rayleigh}
Soit $\MM{A}$ matrice hermitienne et $\VV{x}$ un vecteur non nul, on appelle quotient de Rayleigh $R(\MM{A},\VV{x})$ le scalaire :
\begin{equation}
    R(\MM{A},\VV{x}) = \frac{\VV{x}^{*} \MM{A} \VV{x}}{\VV{x}^{*}\VV{x}}.
\end{equation}
où $\VV{x}^{*}$ désigne le vecteur adjoint de $\VV{x}$, c'est-à-dire le conjugué du vecteur transposé.
\end{definition}

Dans le cas où $\MM{A}$ et $\VV{x}$ sont à coefficients réels, alors $\VV{x}^{*}$ se réduit au vecteur transposé de $\VV{x}$.
%Notons que pour tout nombre réel $c$, $R(A, c x) = R(A,x)$.
%%%%%%%%%%%%%%%%%%
\medskipvm
Le quotient de Rayleigh atteint un minimum $\lambda_{\min}$ (qui n'est autre que la plus petite valeur propre de $\MM{A}$)
lorsque $\VV{x}$ est un vecteur propre $\VV{v}_{\min}$ associé à cette valeur.
%%%%%%%%%%%%%%%%%
\medskipvm
De plus, $\forall \VV{x}$, $R(\MM{A}, \VV{x}) \leq \lambda_{\max}$ (où $\lambda_{\max}$ est la plus grande
valeur propre de $\MM{A}$ de vecteur propre associé $\VV{v}_{\max}$) et $R(\MM{A}, \VV{v}_{\max}) = \lambda_{\max}$.
Ainsi, le quotient de Rayleigh, combiné au théorème du minimax de von Neumann,\index[aut]{Neumann (John, von -), 1903-1957, Hongrois}
permet de déterminer une à une toutes les valeurs propres d'une matrice.

On peut également l'employer pour calculer une valeur approchée d'une valeur propre à partir d'une approximation
d'un vecteur propre.
Ces idées forment d'ailleurs la base de l’algorithme d’itération de Rayleigh.

\medskip
\subsubsection{Méthode itérative de Rayleigh}\index[aut]{Rayleigh (John William Strutt, troisième baron -), 1842-1919, Anglais}

On utilise l'algorithme suivant:
\begin{enumerate}
   \item choix d'un vecteur initial $\VV{x}_i$;
   \item résolution de $\MM{K}\VV{x}_{i+1} = \MM{M} \VV{x}_i$;
   \item test de convergence:
	\begin{itemize}
	   \item si $|\VV{x}_{i+1}-\VV{x}_i|<\varepsilon$, alors aller au point 4;
	   \item sinon retourner au point 1 pour choisir un nouveau vecteur;
	\end{itemize}
   \item $\VV{\varphi}=\VV{x}_{i+1}$ et $\omega^2=\dfrac{\langle \VV{\Phi}, \MM{K}\VV{\Phi}\rangle}{\langle \VV{\Phi}, \MM{M}\VV{\Phi}\rangle}$
\end{enumerate}
\textcolorblue{Ce processus converge vers le mode propre fondamental.}
%%%%%%%%%%%%%%%%%
\medskipvm
Lorsque l'on veut déterminer le mode le plus proche d'une pulsation donnée $\varpi$, il suffit de remplacer
la matrice de rigidité par $\MM{\tilde{K}}=\MM{K} - \varpi^2 \MM{M}$.
%%%%%%%%%%%%%%%%%
\medskipvm
\textcolorblue{Le processus converge alors vers le mode de pulsation $\tilde{\omega}^2=\omega^2-\varpi^2$.}

\medskip
\subsubsection{Itérations des sous-espaces}
La méthode précédente peut être étendue en prenant plusieurs vecteurs initiaux et en se plaçant dans
le sous-espace qu'ils définissent. Les pulsations propres doivent alors être calculées à chaque itération
en calculant tous les modes propres du système réduit au sous-espace étudié.

\medskip
\subsubsection{Sous-structuration: méthode de Craig et Bampton}\index[aut]{Bampton (Mervyn Cyril Charles), ?, Américain}\index[aut]{Craig (Roy R. Jr), ?, Américain}

Considérons une structure comportant $n$ ddl et ayant une matrice masse $\MM{M}$ et une matrice rigidité $\MM{K}$.
L'utilisation de la méthode de Craig-Bampton\index[aut]{Bampton (Mervyn Cyril Charles), ?, Américain}\index[aut]{Craig (Roy R. Jr), ?, Américain}
impose de décomposer les ddl de la structure en deux parties :
\begin{itemize}
   \item les \textcolorblue{ddl <<~frontière~>>}: on considère que ces ddl sont ceux pouvant potentiellement être chargés au cours du temps
	et ceux sur lesquels s'appliquent des conditions limites (encastrement, appui simple...). Les chargements volumiques
	(tel le poids) n'influent pas sur la détermination des degrés de liberté frontière, sans quoi cette décomposition n'aurait
	pas de sens, ces ddl $\VV{q}_L$ (L comme liaison).

	Dans le cas de la sous-structuration, les ddl frontières correspondent trivialement aux ddl aux interfaces entre
	les différentes sous-structures.
   \item les \textcolorblue{ddl <<~intérieurs~>>}: il s'agit de tous les autres degrés de liberté (un chargement volumique peut éventuellement
	s'appliquer sur ces ddl), ces degrés de liberté seront notés $\VV{q}_I$.
\end{itemize}
%%%%%%%%%%%%%%%%%%%
\medskipvm
La base de réduction se compose de deux types de modes.
\begin{itemize}
   \item Les \textcolorblue{modes encastrés}: il s'agit des modes propres de la structure calculés en considérant les ddl frontière encastrés.
   \item Les \textcolorblue{modes statiques}: ces modes sont obtenus en calculant la déformée statique de la structure lorsqu'un ddl frontière
	est imposé à 1, tous les autres étant imposés à 0.
\end{itemize}
%%%%%%%%%%%%%%%%%%%
\medskipvm
\textcolorgreen{Avantages de la méthode:}
\begin{itemize}
   \item elle est facile à programmer
   \item sa stabilité est connue.
   \item elle apporte de bons résultats pour des structures de taille raisonnable.
   \item elle permet d'obtenir les ddl frontière dans le vecteur réduit ce qui peut s'avérer très utile dans le cas de problèmes de contacts
	par exemple.
\end{itemize}
%%%%%%%%%%%%%%%%%%%
\medskipvm
\textcolorred{Inconvénients de la méthode:}
\begin{itemize}
   \item cette méthode n'est pas celle qui permet d'obtenir la meilleure réduction du système et peut donc s'avérer coûteuse en temps de calcul.
   \item sans amortissement structural, la méthode peut diverger au voisinage des fréquences propres de la structure (à cause du gain infini à
	la résonance sans amortissement).
\end{itemize}


%\medskip
%\section{Propagation d'une onde dans un milieu, une structure}




%\medskip
%\section{Interaction d'une onde avec une structure}



% Espace des phases => Poincaré
% Chaos : attracteurd étranges

%\section{Présentation des méthodes sur le cas des chocs large bande}% à mettre en complément à la fin du chapitre


\medskip
\section{Remarques sur l'amortissement}

Vous avez peut-être déjà rencontré des cas pour lesquels ont été développés des modèles vibratoires adaptés au problème à traiter.
En voici quelques exemples:
\begin{itemize}
   \item vibration transversale des cordes;
   \item vibration longitudinale dans les barres;
   \item vibration de torsion dans les barres;
   \item vibration de flexion dans les poutres;
   \item vibration des membranes;
   \item vibration des plaques;
   \item propagation des ondes quasi-longitudinales dans les barres;
   \item propagation des ondes de flexion dans les poutres...
\end{itemize}

\begin{histoire}
Il ne faut pas négliger ces modèles simplifiés. Ne serait-ce que c'est eux qui ont vu le jour en premier...

\medskip
C'est en 1787 à Leipzig que Chladni\index[aut]{Chladni (Ernst Florens Friedrich), 1756-1827, Allemand} 
met en évidence expérimentalement la formation de lignes nodales sur une plaque libre avec du sable.
Wheatstone\index[aut]{Wheatstone (Charles), 1802-1875, Anglais} 
et Rayleigh,\index[aut]{Rayleigh (John William Strutt, troisième baron -), 1842-1919, Anglais} 
respectivement en 1833 et 1873, utiliseront des modes de poutre libre pour essayer
de comprendre et d'expliquer les figures de Chladni.

En 1909, Ritz,\index[aut]{Ritz}{Ritz (Walther), 1878-1909, Suisse} toujours sur ce problème de la plaque libre, utilisera pour 
la première fois la méthode qui porte son nom.
Les premiers résultats concernant la plaque encastré ne viendront qu'en 1931, et sont dus à Sezawa.\index[aut]{Sezawa (Katsutada), ?, Japonais}

En 1939, Igushi développe une méthode pour obtenir certains résultats analytiques, mais les premières
synthèses complètes sur les méthodes utilisables pour calculer les fréquences naturelles et les déformées
modales de plaques ne viendront qu'en 1954 par Warburton,\index[aut]{Warburton (Geoffrey Barratt), 1924-2009, Anglais} et 1969 par 
Leissa.\index[aut]{Leissa (Arthur W.), 1932-, Américain}
\end{histoire}

\medskip
La méthodologie est toujours la même et se base sur la \textcolorblue{technique de séparation des variables} qui
permet de dire que les variables d'espace et de temps peuvent être séparées.
On écrira donc un déplacement transversal $w(x,y)$ comme produit d'une fonction dépendant de l'espace $X(x)$ et
d'une fonction dépendant du temps $T(t)$: $w(x,t)=X(x)T(t)$.
Ainsi, il sera <<~aisé~>> de résoudre le problème (en ayant pris en compte les conditions aux limites évidemment).

\medskip
Généralement, dans un premier temps, lors du développement de ces modèles de cordes, barres, poutres, membranes
et plaques, l'amortissement n'est pas pris en compte.
Les solutions obtenues pour les réponses libres ne présentent donc pas de décroissance de l'amplitude des mouvements
dans le temps.
Il est possible d'intégrer cet amortissement de plusieurs façons:
\begin{itemize}
   \item \textcolorblue{Facteur d'amortissement modal:} il s'agit de la manière la plus simple d'introduire l'amortissement en
	incluant un terme dissipatif correspondant à un modèle d'amortissement visqueux sur la fonction dépendant du temps 
	uniquement. \textcolorgris{Dans ce cas, la fonction de dépendance temporelle $T(t)$, qui avait pour équation différentielle
	$\ddot{T}_n(t)+\omega^2_nT_n(t)=0$, pour les modes $n\ge1$, devra désormais répondre à l'équation
	$\ddot{T}_n(t)+2\xi_n\omega_n\dot{T}_n(t)+\omega^2_nT_n(t)=0$, où $\xi_n$ est le facteur d'amortissement modal};	
   \item \textcolorblue{Coefficient d'amortissement dans l'équation d'onde:}
	un terme dissipatif est introduit directement dans l'équation des ondes. Celui-ci peut être proportionnel au milieu externe dans lequel
	se produit le phénomène \textcolorgris{(par exemple proportionnel à la vitesse de déformation hors plan d'une membrane 
	vibrant dans un fluide, comme l'air)}, ou proportionnel au milieu interne considéré \textcolorgris{(par exemple proportionnel à
	la vitesse de fluctuation des contraintes dans une poutre: modèle de Kelvin-Voigt)};
   \item \textcolorblue{Dissipation aux limites:} l'amortissement peut également être introduit dans la définition des conditions
	aux limites, par exemple pour prendre en compte le mode de fixation de la structure. \textcolorgris{(Par exemple,
	dans le cas des ondes longitudinales dans une barre, on introduit un ressort et un amortisseur à chaque bout de
	la barre).}
\end{itemize}

Ces remarques, bien que générales, nécessitent d'être correctement prises en compte pour ces modèles simplifiés.


\medskip
\section{Pour aller plus loin: cas des chocs large bande}

\textcolorgreen{On s'intéresse au cas des chocs, car il constituent un cas plus compliqué que
de la <<~dynamique lente~>>, sans rien enlever en généralité aux méthodes.}

\medskip
\textcolorred{En dynamique transitoire, lorsque le contenu fréquentiel de la sollicitation est large,
on montre que les erreurs numériques faites sur chaque longueur d'onde se cumulent, d'où une
dégradation de la qualité attendue du résultat plus rapide que prévue.}
De plus, une erreur sur la périodicité des oscillations due à un trop faible nombre de pas de temps et
un déphasage des oscillations à cause de cette accumulation des erreurs numériques au cours des pas
de temps peuvent être observées.

Dans de tels cas, il est possible de \textcolorblue{se placer dans le domaine fréquentiel}\index{Domaine fréquentiel}
(à l'aide de la FFT),\index{Transformée de Fourier}\index{FFT}  ce qui conduit à résoudre un problème de vibrations forcées sur une très large bande
de fréquences incluant à la fois les basses et les moyennes fréquences pour l'étude des chocs.
La solution temporelle est ensuite reconstruite par transformée de Fourier\index[aut]{Fourier (Jean Baptiste Joseph), 1768-1830, Français}\index{Transformée de Fourier}
inverse.

Les deux domaines fréquentiels ayant des propriétés différentes, on recourt à des \textcolorblue{outils
de résolution différents.}

\medskip
Dans le domaine des \textbf{basses fréquences}, les phénomènes vibratoires générés par l'excitation ont une longueur
d'onde grande face à la structure donc uniquement quelques oscillations sont observables. De plus
la structure présente un comportement modal (modes bien distincts les uns des autres).
La modélisation est maîtrisée : \textcolorblue{éléments finis} sur base modale, complété si besoin de modes
statiques.

\medskip
Pour les \textbf{hautes fréquences}, les longueurs d'ondes sont petites et une centaine d'oscillations est présente sur une
dimension de la structure. Il n'est pas approprié de regarder les grandeurs locales, mais plutôt les
grandeurs moyennées en espace et en fréquence. On utilise généralement la \textcolorblue{SEA} \index{SEA}
qui donne un niveau énergétique vibratoire moyen par sous-structure. Cette méthode ne permet
pas d'obtenir une solution prédictive car elle requiert la connaissance à priori de facteurs de
couplages mesurés.

\medskip
En \textbf{moyennes fréquences}, plusieurs dizaines d'oscillations apparaissent sur une dimension de la structure, et la
déformée est très sensible aux conditions aux limites et aux paramètres matériaux
de la structure. Si un comportement modal est encore visible, les modes sont moins bien
séparés (par exemple plusieurs modes présents par Hertz, ces modes étant couplés
par l'amortissement) : la MEF est mal adaptée à cause du raffinement de maillage
nécessaire, et le calcul de la base modal est également hors de portée. Les méthodes
énergétiques quant à elles sont trop globales et ne permettent pas une description satisfaisante
de la solution.

Si la structure est divisible en sous-structures homogènes, on peut utiliser la \textcolorblue{Théorie
Variationnelle des Rayons Complexes (TVRC)}\index{Thérorie Variationnelle des Rayons Complexes} introduite
par Ladevèze en 1996:\index[aut]{Ladevèze (Pierre), ?- , Français}
les conditions de continuité en déplacements et en efforts aux interfaces entre
sous-structures n'ont pas besoin d'être vérifiées à priori, mais uniquement au sens faible par une
formulation variationnelle.

La TVRC permet l'utilisation d'approximations indépendantes par sous-structure.
La solution est supposée bien décrite par la superposition d'un nombre infini de modes locaux, appelés rayons, issus de la vérification des équations d'équilibre dynamique et des relations de comportement par sous-structure. Ces rayons sont à deux échelles: une lente et une rapide. L'échelle rapide est traitée analytiquement (sinon coût numérique élevé), et l'échelle lente numériquement, car elle conduit à un problème à faible nombre d'inconnues.

On profitera de la rapide dispersion des modes moyennes fréquences dans les milieux dispersifs amortis ainsi que de la version large bande de la TVRC développée en 2004 et 2005~\cite{Lit-Chevreuil}\index[aut]{Chevreuil (Mathilde), ?- , Française}

\medskip
\subsection{Approches temporelles}\index{Domaine temporel}

\subsubsection{Discrétisation spatiale}
Une discrétisation par la MEF est mal adaptée aux phénomènes à fort gradient tels
que les chocs, car il faut soit un maillage très fin, soit une interpolation par des polynômes
de degré élevé, ce qui dans les deux cas augmente considérablement l'effort de calcul.

Étant donné le caractère très localisé des ondes propagatives en dynamique transitoire,
la \textcolorblue{méthode des éléments finis adaptatifs} répond au besoin d'enrichir le modèle
localement en raffinant le maillage uniquement sur les fronts d'onde de manière contrôlée et
automatique.

Ces méthodes recourent à un estimateur d'erreur à priori :
\begin{itemize}
	\item estimateur construit sur les résidus d'équilibre pour l'adaptation de
		maillage dans le cadre de la propagation d'ondes;
	\item estimateur utilisant le lissage des contraintes;
	\item estimateur basé sur l'erreur en relation de comportement.
\end{itemize}

\medskip
\subsubsection{Décomposition de domaine en dynamique transitoire}\label{Sec-Schur}

Le domaine est décomposé en sous-domaines plus petits à calculer.
On pourra se servir de la parallélisation du problème.
\textcolorblue{Le problème est donc condensé sur les quantités d'interface entre sous-domaines,
ce qui conduit à un problème de taille réduite.}

La plupart des méthodes utilisées sont des méthodes sans recouvrement; elles
peuvent être primales, duales ou mixtes. Le problème d'interface est résolu
de façon itérative, ce qui évite la construction explicite du complément
de Schur\footnote{%
En algèbre linéaire et plus précisément en théorie des matrices, le complément de
Schur\index{Complément de Schur}\index[aut]{Schur (Issai), 1875-1941, Russe} est défini comme suit. Soit :
\begin{equation} \MM{M}=\MM*{\MM{A} & \MM{B} \\ \MM{C} & \MM{D}} \end{equation}
une matrice de dimension $(p+q)\times(p+q)$, où les blocs $\MM{A}$, $\MM{B}$, $\MM{C}$, $\MM{D}$ sont des matrices
de dimensions respectives $p\times p$, $p\times q$, $q\times p$ et $q\times q$, avec $\MM{D}$ inversible.
Alors, le complément de Schur du bloc $\MM{D}$ de la matrice $\MM{M}$ est constitué par la matrice
de dimension $p\times p$ suivante :
\begin{equation} \MM{A} - \MM{B}\cdot \MMI{D}\cdot \MM{C}\end{equation}

Lorsque $\MM{B}$ est la transposée de $\MM{C}$, la matrice $\MM{M}$ est symétrique définie-positive
si et seulement si $\MM{D}$ et son complément de Schur dans $\MM{M}$ le sont.
}, mais nécessite un taux de convergence élevé pour être
efficace.

Dans la méthode duale, les efforts sont privilégiés : la méthode propose
à priori des efforts en équilibre aux interface et cherche à écrire la continuité
en déplacements. L'inconnue principale, i.e. les inter-efforts entre sous-structures,
sont les multiplicateurs de Lagrange aux interfaces.\index[aut]{Lagrange (Joseph Louis, comte de -), 1736-1813, Italien}\index{Multiplicateurs de Lagrange}

\medskip
\subsubsection{Discrétisation temporelle}

Les méthodes d'intégration directe sont nombreuses et mieux adaptées que les
techniques de bases réduites pour les chocs relativement rapides qui mettent en jeu des
fréquences élevées.

Notons qu'il existe des méthodes qui s'affranchissent de la discrétisation temporelle et
s'appuient sur une méthode asymptotique numérique pour déterminer la réponse
transitoire de la structure; ces méthodes demandent encore à être
développées pour les variations temporelles rapides comme les chocs.

Parmis les méthodes d'intégration directes, on utilise classiquement :
\begin{itemize}
	\item les \textcolorblue{schémas de Newmark}\index[aut]{Newmark (Nathan Mortimore), 1910-1981, Américain}\index{Schéma! de Newmark} (voir chapitre précédent) pour une intégration d'ordre 2 :
		les schémas précis au second ordre des différences centrées et de
		l'accélération moyenne sont privilégiés pour les faibles erreurs d'amplitude
		et de périodicité qu'ils engendrent.
		
		Le schéma des différences centrées\index{Schéma! des différences finies centrées}
		est explicite et adapté pour les chargements de dynamique rapide et les
		problèmes non linéaires (car la matrice dynamique à inverser est diagonale),
		mais nécessite de vérifier que le signal ne se propage pas de plus d'un élément
		pendant un pas de temps (condition de Courant).\index[aut]{Courant (Richard), 1888-1972, Américain}
		
		Le schéma de l'accélération moyenne est implicite mais inconditionnellement
		stable, bien adapté pour les chargement peu rapides.
	\item la \textcolorblue{méthode de Galerkine discontinue}:\index[aut]{Galerkine (Boris), 1871-1945, Russe}\index{Méthode de Galerkine! discontinue}
		Elle autorise les variables du problème déplacement et vitesse à être
		discontinues en temps.
		À l'ordre zéro (champs constants sur chaque intervalle de temps), elle permet
		de s'affranchir des oscillations numériques occasionnées lors du traitement
		d'un front d'onde.
		Toutefois, ce schéma dissipe énormément et demande une discrétisation
		très fine pour bien représenter les irrégularités.
	\item la \textcolorblue{TXFEM (Time eXtended FEM)}:\index{TXFEM}
		Elle utilise une base de fonctions de forme en temps enrichie formant une partition de l'unité.
		Le schéma est équivalent à certaines méthodes de Galerkine discontinues, le nombre de
		pas de temps inférieur à Newmark, et les oscillations numériques atténuées.
		Elle est bien adaptée pour le traitement des discontinuités en temps et
		notamment les chocs. Voir le paragraphe \ref{Sec-XFEM} pour une courte description.
\end{itemize}

\medskip
\subsection{Approches fréquentielles}\index{Domaine fréquentiel}

Afin de s'affranchir de l'intégration temporelle et des soucis numériques associés, il est
possible de réécrire le problème temporelle en un problème fréquentiel (grâce à
la FFT).\index{Transformée de Fourier}\index{FFT}

\textcolorblue{L'approche fréquentielle est également plus adaptée dans les situations pour
lesquelles des paramètres physiques dépendent de la fréquence.}\index{Domaine fréquentiel}

\medskip
En appliquant la transformée de Fourier\index[aut]{Fourier (Jean Baptiste Joseph), 1768-1830, Français}\index{Transformée de Fourier}
à toutes les quantités dépendant du temps,
on obtient des quantités qui dépendent de la fréquence.
Ce faisant, le problème à résoudre devient un problème de vibration
forcée sur une bande de fréquence.
Il faut alors calculer des fonctions de réponse en fréquence (FRF)\index{FRF} sur une large plage
de fréquences.

\medskip
\subsubsection{Cas des basses fréquences}

Compte tenu de la grande taille des longueurs d'ondes et du fait que l'on a peu de modes, bien séparés,
les méthodes utilisées sont basées sur les \textcolorblue{éléments finis}.

En réécrivant le problème dynamique dans le cas d'une sollicitation harmonique de pulsation
$\omega$, il vient (comme nous venons de le présenter avant):
\begin{equation}
\left(-\omega^2\MM{M}+i\omega \MM{C}+\MM{K}\right)\VV{q} = \VV{F}
\end{equation}

En utilisant \textcolorblue{l'hypothèse de Basile}\footnote{%
Si le seul amortissement entrant en jeu est un amortissement structurel (dissipation
interne du matériau pour une structure homogène), il est alors licite de faire
l'hypothèse d'un amortissement proportionnel, encore appelé hypothèse de
Basile. Dans ce cas $\MM{C}$ s'exprime comme combinaison linéaire de $\MM{M}$ et $\MM{K}$,
et sa projection sur les modes propres est diagonale.
} sur l'amortissement, il est possible de calculer les premiers modes propres\index{Mode propre}
associés aux plus petites fréquences propre\index{Fréquence propre} du système.
La solution approchée est alors projetée sur les sous-espaces propres\index{Espace!propre} associés.
On résoud alors un système diagonale de petite taille.

\medskip
\subsubsection{Cas des moyennes fréquences}
En moyennes fréquences, l'approche précédente (EF) nécessiterait d'utiliser une grande quantité de polynômes
à cause du caractère très oscillant, ou à augmenter le degré d'interpolation. De plus, il faut
prendre en compte plus de modes, qui sont de moins en moins bien séparés.

\medskip
Notons que l'on peut minimiser l'influence du raffinage du maillage en localisant celui-ci uniquement
où la dynamique locale le demande. Pour cela, des estimateurs d'erreur à posteriori ont été
développés pour les structures, mais également pour l'acoustique.

La dispersion numérique est moindre avec les éléments finis stabilisés :  tels que les \textcolorblue{Galerkin
Least Squares} et \textcolorblue{Galerkin Gradient Least Squares}\index{Méthode de Galerkine! Least Squares}\index{Méthode de Galerkine! Gradient Least Squares}...
qui n'intervienne pas sur la forme variationnelle mais sur les matrices issues de celle-ci.

\medskip
On peut également essayer de diminuer la taille de la base modale à prendre en compte en
ne retenant que les modes propres qui maximisent l'opérateur d'excitabilité; mais pour cela
il faut d'abord calculer la base complète...

On peut également utiliser un autre espace de projection que les modes propres classiques :
par exemple les premiers modes d'un opérateur d'énergie relatif à une bande de fréquence
(Soize 1998).\index[aut]{Soize (Christian), 1948- , Français}
Cette approche peut être couplée à la \textcolorblue{théorie des structures floues}\index{Thérorie des structures floues}
(Soize 86)  pour prendre en compte la complexité de la structure de manière probabiliste.

\medskip
On peut également sous-structurer le domaine.
Dans la \textcolorblue{Component Modal Synthesis},\index{Component Modal Synthesis} les
modes propres\index{Mode propre} de chaque sous-structures  servent de base pour la solution
de la structure entière.

\bigskip
Une <<~deuxième~>> approche consiste à utiliser des \textcolorblue{éléments enrichis}\index{Elément enrichi}
(voir encore une fois le chapitre \ref{Ch-XFEM}).
Ces éléments sont développés pour pouvoir prendre en compte le caractère
oscillant de la solution  en enrichissant les fonctions de base utilisées afin de pouvoir
mieux reproduire la solution.

\medskip
Les \textcolorblue{éléments finis hiérarchiques,\index{EF hiérarchique} issus des méthodes p}\index{Méthode p}
(voir le paragraphe \ref{Sec-rhp}:
\textcolorgreen{l'augmentation du degré polynomial des fonctions de forme peut se voir comme
une substitution à un raffinement de maillage, mais les estimateurs d'erreur des p-methods sont meilleurs
que ceux des h-methods}),\index{Méthode h}\index{Méthode p} permettent une réutilisation à l'ordre $p+1$ des matrices
de masse et de raideur élémentaires issues de l'ordre $p$.

Les \textcolorblue{éléments finis multiéchelles} (voir paragraphe \ref{Sec-ssstruc}) où la solution est
recherchée comme somme d'une composante calculable à l'échelle grossière en espace et
d'une composante non calculable associée à l'échelle fine.

La \textcolorblue{Méthode de partition de l'unité (PUM)}\index{Méthode de partition de l'unité} (voir paragraphe \ref{Sec-partition}) utilise un recouvrement
du domaine initial en un ensemble  de maillages, chacun étant enrichi et vérifiant la partition de l'unité.
La FEM associée à la PUM donne naissance à deux grandes familles d'approches :
la G-FEM (Generalized FEM)\index{G-FEM} et la X-FEM (eXtended FEM),\index{X-FEM}\index{Méthode des éléments finis étendue}
voir paragraphe \ref{Sec-XFEM}.
Si les fonctions d'enrichissement ne sont pas activées, on se retrouve avec la FEM.

la \textcolorblue{Méthode d'enrichissement discontinu}\index{Méthode d'enrichissement discontinu}
(Discontinuous Enrichment Method) est une méthode de Galerkine éléments finis
discontinue avec multiplicateurs de Lagrange\index{Méthode de Galerkine! discontinue avec multiplicateurs de Lagrange}
dédiée aux application de forts gradients ou oscillations rapides.

\bigskip
La \textcolorblue{BEM}\index{BEM}\index{Méthode des éléments finis frontière}
(voir paragraphe \ref{Sec-BEM}) est encore une <<~troisième~>> approche.
Seuls les bords sont maillés afin de réduire le nombre de ddl.
La formulation intégrale de la frontière établit un lien entre les champs intérieurs
et les quantités sur les bords.
La matrice obtenue est petite, mais pleine et non symétrique.

\bigskip
Les \textcolorblue{méthodes sans maillage}\index{Méthode sans maillage} (voir paragraphe \ref{Sec-meshless}):
Dans la EFGM\index{EFGM} (Element Free Galerkin Method ou méthode de Galerkine sans maillage),\index{Méthode de Galerkine! sans maillage}
on n'a plus qu'un nuage de points sans connectivité entre eux.
On utilise des fonctions de formes construites selon la méthode des moindres carrés mobiles et
formant une partition de l'unité,\index{Méthode de partition de l'unité} qui peuvent être polynomiales
ou sinusoïdales.
Néanmoins, comme elle est basée sur une discrétisation nodale, elle nécessite également
un grand nombre de nœuds aux fréquences plus élevées.

\bigskip
Les \textcolorblue{méthodes des éléments discontinus}:\index{Méthodes des éléments discontinus}
Pour des géométries simples (utilisation en construction navale), on utilise des solutions
analytiques ou semi-analytiques sur des sous-domaines simples (poutres, plaques rectangulaires...)
pour construire la structure complète.
Lorsque cette méthode est applicable \textcolorblue{elle donne de bon résultats aussi bien en basses fréquences, en moyennes fréquences et en hautes fréquences}.

\bigskip
Les \textcolorblue{méthodes de Trefftz}:\index{Méthode de Trefftz}\index[aut]{Trefftz (Erich Immanuel), 1888-1937, Allemand}
Elles utilisent des fonctions de base définies sur tout le domaine de la sous-structure considérée
et vérifiant exactement l'équation dynamique et la loi de comportement : la solution est représentée
par la superposition de ces fonctions; mais il faut encore vérifier les conditions aux limites et de
transmission.
Les matrices sont de petite taille mais très mal conditionnées.

Les T-éléments lient la démarche Trefftz\index[aut]{Trefftz (Erich Immanuel), 1888-1937, Allemand} et la FEM:
Trefftz au sein de chaque élément..

Pour les problèmes de \textcolorblue{vibroacoustique} la \textcolorblue{WBT (Wave Based Technique)}\index{WBT}
a été développée:
la structure n'est pas discrétisée comme pour les T-éléments mais est décomposée en
éléments de grande taille par rapport à la dimension de la structure. les fonctions de base
particulières utilisées améliorent le conditionnement des matrices, ce qui conduit à des
matrices de petite taille pleines et non symétriques. L'intégration sur les bords coûte cher
en moyennes fréquences et le conditionnement de la matrice en hautes fréquences se dégrade du fait de la discrétisation des amplitudes
(seules certaines directions de propagation sont prises en compte).

\medskip
\subsubsection{Cas de hautes fréquences}
Comme dit précédemment, on ne représente pas la solution localement mais on ne
s'intéresse qu'à des grandeurs moyennées.

\bigskip
La \textcolorblue{SEA\index{SEA} (Statistical Energy Analysis) est la méthode de référence pour les hautes fréquences}.
La structure est découpée en sous-structures.
Ensuite des regroupements de modes sont construits tels que statistiquement le niveau
de chacun des groupes de modes soit semblable : la méthode repose donc sur l'hypothèse
d'une forte densité modale dans la bande de fréquence étudiée.
Chaque groupe de mode est associé à un ddl: le problème à résoudre issu
de l'équilibre énergétique est par conséquent à faible nombre de ddl.
Cet équilibre se traduit par un bilan de puissance dans lequel la puissance injectée
à une sous-structure par des forces extérieures, aléatoires et stationnaires sur de
larges bandes de fréquences, est égale à la somme de la puissance dissipée
dans cette sous-structure par amortissement et la puissance transmise à l'ensemble
des sous-structures voisines avec lesquelles elle est connectée, appelée puissance
de couplage.
\textcolorred{L'hypothèse forte de la SEA\index{SEA} concerne cette puissance de couplage entre
deux sous-structures qui est supposée proportionnelle à la différence de leurs énergies par mode,\index{Mode propre}
le facteur de proportionnalité étant le coefficient de couplage.}

La SEA\index{SEA} est parfaitement adaptée aux hautes fréquences, mais trop globale et trop imprécise pour
décrire finement le comportement en moyennes fréquences.

De plus les coefficients de couplages ne sont connus explicitement à priori que pour
des géométries très particulières et nécessitent donc en général de recourir
à des expériences, ce qui fait de la SEA\index{SEA} une méthode non prédictive.

Des stratégies de calcul de ces coefficients de couplage existent, mais pour certains
régimes d'excitation la notion même de coefficient de couplage n'est plus réaliste.

\bigskip
La \textcolorblue{méthode de diffusion d'énergie}:\index{Méthode de diffusion d'énergie}
Elle apporte un effet local à la SEA\index{SEA} en décrivant de manière continue les variables
énergétiques. Elle a été appliquée à des cas simples et l'analogie avec la thermique
n'est pas démontrée pour des sollicitations et des géométries quelconques.

\bigskip
L'\textcolorblue{analyse ondulatoire de l'énergie}:\index{Analyse ondulatoire de l'énergie}
Elle généralise la SEA\index{SEA} en ce qu'elle considère le champ d'ondes non plus diffus mais
directionnel en introduisant un champ d'ondes aléatoires propagatives dans les
sous-structures et des coefficients de couplages qui varient selon les angles
d'incidence aux interfaces.

\bigskip
Les \textcolorblue{MES (Méthodes Énergétiques Simplifiées)}:\index{Méthodes Énergétiques Simplifiées}
Elles se proposent de pallier les insuffisances de la méthode de diffusion de l'énergie
en donnant une représentation locale des phénomènes. Le bilan de puissance est
écrit aussi bien à l'intérieur des sous-structures que de leurs couplages.
La connaissance des coefficients de couplages à priori demeure un problème.

\bigskip
D'autres méthodes existent encore : développements asymptotiques,
méthode de l'enveloppe, méthode des chemins structuraux.

La \textcolorblue{méthode des rayons}:\index{Méthode des rayons}
consiste à suivre les rayons vibratoires le long
de leur parcours par l'étude de leur propagation, réflexion et transmission
entre sous-structures par les lois de Snell-Descartes\index[aut]{Snell (Willebrord Snell van - ou Snellius), 1580-1626, Néerlandais}\index[aut]{Descartes (René), 1596-1650, Français}
de l'optique géométrique jusqu'à l'amortissement des ondes.
La RTM permet de connaître la direction privilégiée de transfert et de
répartition spatiale de l'énergie, mais à un coût numérique très
élevé. De plus, les coefficients de transmissions ne sont pas connus à priori...

\medskip
\subsection{Remarques}
Dans l'approche fréquentielle, il faut déterminer les fréquences déterminant
la jonction BF/MF et MF/HF.

La fréquence BF/MF agit sur le coût de calcul : elle doit être la plus grande possible, mais
telle qu'à partir de cette fréquence, les modes de la structure deviennent locaux,
tout en conservant la séparation des modes (en pratique entre 300 et 600 Hz).

Le choix de la fréquence MF/HF joue sur la qualité de la vitesse calculée
et donc sur l'énergie cinétique qui sert pour restaurer la réponse
temporelle. Cette fréquence doit être au moins égale à $1/T$ où
$T$ est la durée du choc d'entrée (si $T=1$ms, alors MF/HF = 2000Hz mini).

la MEF utilisée en basses fréquences doit utiliser les $n$ premiers modes pour la base réduite avec $n$ tel que la fréquence de ce mode soit de $2\times$ la fréquence BF/MF. On utilisera également la règle classique d'\textcolorblue{au moins 10 éléments linéaires par longueur d'onde pour le maillage}.

La FFT\index{Transformée de Fourier}\index{FFT} requiert par ailleurs que le chargement soit périodique. Le temps correspondant à cette période, $T_0$, doit être choisi judicieusement. Dans le cas d'un choc, on a donc $T_0 > T$, mais il faut également le choisir tel que la réponse transitoire de la structure s'éteigne avant la fin de cet intervalle de temps. De plus, ce temps influe sur l'échantillonnage fréquentiel du calcul de la FRF. Les pulsations pour lesquelles la FRF est calculée doivent être telles que $\omega_n=2\pi n f_0$ avec $f_0=1/T_0$. Il faut également que le nombre de pas de fréquences $N$ soit une puissance de 2 pour utiliser la FFT\index{Transformée de Fourier}\index{FFT} et son efficacité, et $N$ soit être tel que $N/T_0 \ge 2 f_{\max}$ avec $f_{\max}$ la fréquence maximale contenue dans le signal.

\medskip
Le choix judicieux de $T_0$ influe directement sur la reconstruction temporelle de la réponse. Pour les structures peu amorties ou pour des chargements longs, $T_0$ est grand, et donc la FFT\index{Transformée de Fourier}\index{FFT} coûteuse. Dans ces cas, des méthodes ont été développées : les fonctions de Green;\index[aut]{Green (George), 1793-1841, Anglais} la Implicit Fourier Transform; et l'amortissement artificiel.


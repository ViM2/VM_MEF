\ifVersionAvecExemplesSepares
  \chapter{Calcul des éléments finisforts équivalents pour une poutre}
  \begin{abstract}
  Dans ce chapitre, nous présentons un petit calcul typique de l'approche mécanicienne.

  Il s'agit d'un cas simple et classique, mais qui est traité, pour une fois, sous sa forme la
  plus générale.
  \end{abstract}
\else
  \section{Exemple: calcul des éléments finisforts équivalents pour une poutre}

  Dans ce paragraphe, et pour poursuivre notre illustration des différentes modélisation d'un même problème, nous présentons 
  un petit calcul de RdM traité selon l'approche mécanicienne.

  Il permet de montrer comment sont calculés les efforts équivalents, dans un cas général, sur une poutre, mais le
  même type de calcul resterait valable pour tout type d'élément.
\fi

\medskip
\ifVersionAvecExemplesSepares
  \section{Problème}
\else
  \subsection{Problème}
\fi

\medskip
Considérons une poutre définie dans son système de coordonnées locales comme montré à la figure 
ci-dessous:
%Let us consider a member defined in its local coordinates system, as shown
%above in \fig{beam1}.
%
%
%\begin{figure}[ht]
 \begin{center}
 \begin{picture}(250,60)(-10,-35)
 %F,F1,F2
  \thicklines
  \put(100,25){P(x)}
  \put(0,25){\vector(0,-1){25}}
  \put(200,25){\vector(0,-1){25}}
  \put(-15,25){F}
  \put(-8,20){1}
  \put(205,25){F}
  \put(212,20){2}

 %C1,C2
  \put(0,0){\oval(20,20)[lt]}
  \put(0,0){\oval(20,20)[rt]}
  \put(0,0){\oval(20,20)[rb]}
  \put(0,-10){\vector(-1,0){6}}
  \put(10,-15){C}
  \put(17,-20){1}
  \put(200,0){\oval(20,20)[lt]}
  \put(200,0){\oval(20,20)[rt]}
  \put(200,0){\oval(20,20)[rb]}
  \put(200,-10){\vector(-1,0){6}}
  \put(180,-15){C}
  \put(187,-20){2}

 %axes
  \thinlines
  \put(0,-15){\line(0,1){30}}
  \put(0,0){\vector(0,-1){35}}
  \put(-10,-35){y}
  \put(0,0){\vector(1,0){235}}
  \put(230,-10){x}

 %Load
  \bezier{180}(0,10)(30,25)(60,15)
  \bezier{240}(60,15)(100,5)(140,20)
  \bezier{180}(140,20)(170,23)(200,18)

 %encastrement
  \multiput(-4,-20)(0,8){5}{/}
  \multiput(200,-20)(0,8){5}{/}
  \put(200,-22){\line(0,1){40}}

 %beam
  \linethickness{3pt}
  \put(0,0){\line(1,0){200}}
 \end{picture}
 \end{center}
% \label{beam1}
% \caption{Considered beam and its local coordinates system}
%\end{figure}

La première extrémité est~$\left( 0,0,0\right)~$ et la seconde~$\left( l,0,0\right)~$, 
où~$l$ est la longueur de la poutre.

Nous supposerons la poutre encastrées aux deux bouts, et soumise à une charge
distribuée définie par la fonction~$P(x)$.

Le problème est plan de sorte que~$\overrightarrow{P(x)}
=P(x)\overrightarrow{y}$

Nous souhaitons déterminer les forces et moments équivalents~$F_1$, $F_2$ et
$C_1$, $C_2$ aux deux extrémités également appelées nœuds 1 et 2.

\medskip
Une analyse statique conduit à:

\begin{quotation}
\begin{equation}
  \label{staticF}
  F_1+F_2+\int_0^l P(x)dx=0 
\end{equation}

\begin{equation}
  \label{staticC}
  C_1+C_2+F_2l+\dint_0^l x P(x)dx=0 
\end{equation}
\end{quotation}


\medskip
Le moment de flexion s'exprime:

\begin{quotation}
$M_z(x)=C_1-F_1x-\dint_0^x x P(x)dx$, ou

$M_z(x)=-C_2-F_2(l-x)-\dint_x^l(v-x) P(x)dv$
\end{quotation}

\medskip
L'énergie interne de la poutre s'esprime:
%The beam's internal energy can be explained as follows~:

%\begin{quote}
\[
\delta J=\frac 12\dint_0^l\dfrac{M_z^2(x)}{EI}dx
\]
%\end{quote}

\medskip
\ifVersionAvecExemplesSepares
  \section{Théorème de Menabrea}\index[aut]{Menabrea (Luigi Federico; général comte -; premier marquis de Valdora), 1867-1869, Italien}\index{théorème!de Menabrea}
\else
  \subsection{Théorème de Menabrea}\index[aut]{Menabrea (Luigi Federico; général comte -; premier marquis de Valdora), 1867-1869, Italien}\index{théorème!de Menabrea}
\fi

Le théorème de Menabrea permet d'écrire:

\begin{quotation}
 ~$\dfrac{\partial \delta J}{\partial C_1} =
  \dfrac{\partial \delta J}{\partial C_2} =
  \dfrac{\partial \delta J}{\partial F_1} =
  \dfrac{\partial \delta J}{\partial F_2} = 0$
\end{quotation}

Ces calculs seront faits après avoir introduit les notations suivantes:
\begin{quotation}\colorblue
 ~$I_0\intervalle{a}{b}=\dint_a^b P(x)dx$

 ~$I_1\intervalle{a}{b}=\dint_a^b x P(x)dx$
\end{quotation}\colorblack


$\dfrac{\partial \delta J}{\partial C_1}=0$~:

\begin{quotation}
$\qquad \dint_0^lM_z(x)dx=0$

$\qquad \dint_0^lC_1-F_1x-I_1(0,x)dx=0$
\end{quotation}

\begin{equation}
  \label{MenaC}
  C_1l-\frac 12F_1l^2-\int_0^lI_1(0,x)dx=0
\end{equation}

$\dfrac{\partial \delta J}{\partial F_1}=0$~:

\begin{quotation}
$\qquad \dint_0^lxM_z(x)dx=0$

$\qquad \dint_0^lC_1-F_1x-I_1(0,x)dx=0$
\end{quotation}

\begin{equation}
  \label{MenaF}
  C_1l-\frac 12F_1l^2-\int_0^lI_1(0,x)dx=0
\end{equation}

Nous utilisons les propriétés des intégrales précédemment définies:

\begin{quotation}
$I\intervalle{a}{b}=I(a,c)+I(c,b)$

$I(b,a)=-I\intervalle{a}{b}$
\end{quotation}

for both~$I_0$ and~$I_1$.


\medskip
\ifVersionAvecExemplesSepares
  \section{Résolution des équations}
\else
  \subsection{Résolution des équations}
\fi

En utilisant ($\ref{MenaC}$) et ($\ref{MenaF}$), on trouve~$F_1$ and~$C_1$,
Puis, en utilisant ($\ref{staticF}$) et ($\ref{staticC}$) on obtient~$F_2$ and~$C_2$.

Les efforts sont:

\begin{quotation}
$F_1=\dfrac 6{l^2}\left( \dint_0^lI_1(0,x)dx-\frac
2l\dint_0^lxI_1(0,x)dx\right)~$

$F_2=-I_0(0,l)-F_1$

$C_1=\dfrac 2l\left( 2\dint_0^lI_1(0,x)dx-\dfrac
3l\dint_0^lxI_1(0,x)dx\right)~$

$C_2=\dfrac 2l\left( \dint_0^lI_1(0,x)dx-\dfrac
3l\dint_0^lxI_1(0,x)dx\right) +lI_0(0,l)-I_1(0,l)$
\end{quotation}

\medskip
\ifVersionAvecExemplesSepares
  \section{Conclusion}
\else
  \subsection{Conclusion}
\fi

Ces forces et moments correspondent à des éléments finisforts de réactions.
Les forces équivalentes utilisées en MEF correspondent aux efforts internes
développés dans la poutre. Ce sont:

\begin{tabbing}
  at node iiiiii\=Forceqqq\= \kill
       \>Force \>Moment\\
  Au nœud 1 \> ~$-F1$ \> ~$-C1$\\
  Au nœud 2 \> ~$-F2$ \> ~$-C2$\\
\end{tabbing}

Analytiquement:

\begin{equation}\colorblue
\label{Feq}\left\{
\begin{array}{l}
  F_1=-\dfrac 6{l^2}\PP{\dint_0^lI_1(0,x)dx-\dfrac 2l\dint_0^lxI_1(0,x)dx} \\
  F_2=I_0(0,l)-F_1 \\
  C_1=-\dfrac 2l\PP{2\dint_0^lI_1(0,x)dx-\dfrac3l\dint_0^lxI_1(0,x)dx} \\
  C_2=-\dfrac 2l\PP{\dint_0^lI_1(0,x)dx-\dfrac3l\dint_0^lxI_1(0,x)dx}
  -lI_0(0,l)+I_1(0,l)
\end{array}
\right.
\end{equation}\colorblack
et sous forme vectorielle:

\begin{quotation}
$\left\{
\begin{array}{l}
  \overrightarrow{F_1}=F_1\overrightarrow{y} \\
  \overrightarrow{F_2}=F_2\overrightarrow{y} \\
  \overrightarrow{C_1}=C_1\overrightarrow{z} \\
  \overrightarrow{C_2}=C_2\overrightarrow{z}
\end{array}
\right.~$
\end{quotation}



\chapter*{Conclusion}\label{Ch-Ccl}
%\addcontentsline{toc}{chapter}{Conclusion}
\markboth{Conclusion}{}


En guise de conclusion à ce document (qui se poursuit par des exemples), nous souhaitions synthétiser
les problèmes qui peuvent survenir lors d'un calcul et ouvrir sur quelques perspectives.



\medskip
\section*{Sur la fiabilité des résultats}
\addcontentsline{toc}{section}{Sur la fiabilité des résultats}

La fiabilité d'un résultat dépend de celle de toute la chaîne d'approximation réalisées depuis la modélisation
du phénomènes physique jusqu'à l'obtention des résultats numériques fournis par le programme de calcul.
Nombre de ces problèmes ont été mentionnés au cours de ce document. Rappelons les ici:
\begin{itemize}
   \item \textcolorblue{erreurs de modélisation}: aussi bien au niveau du choix des équations mathématiques décrivant le phénomène, 
	que de la représentativité des conditions aux limites choisies;
   \item \textcolorblue{erreurs de discrétisation}: elles sont liées aux choix des méthodes numériques (MEF ou autre), aux problèmes
	d'intégration, de représentation du domaine...
   \item \textcolorblue{erreurs de verrouillage numérique}: elles concernent des problèmes survenant lors du traitement de paramètres
	introduits dans le calcul tels que les pas de temps, des modes parasites, des instabilités numériques...
   \item \textcolorblue{incertitudes des données}: connaissance approximative des lois de comportement, des efforts,
	des liaisons... Il faut alors procéder à une approche fiabiliste qui n'a pas été présentée dans ce document.
   \item \textcolorblue{erreurs d'arrondis}: la manière dont un ordinateur traite un nombre est soumise à des contraintes
	(représentation en base 2 par exemple, approximation à $n$ décimales...). Ces erreurs, mêmes infimes, peuvent
	dans certains cas se cumuler pour aboutir à un résultat faux.
\end{itemize}

Lorsque cela est possible, on n'hésitera donc pas a effectuer des comparaison avec des essais, ce que l'on nomme
\textcolorblue{recalage calcul-essais}.



\medskip
\section*{Quelques perspectives}
\addcontentsline{toc}{section}{Quelques perspectives}

Le calcul scientifique d'une manière générale, et la MEF en particulier, sont utilisés de manière intensive dans
tous les secteurs.

\medskip
Toute amélioration de la performance des méthodes numériques est donc un enjeu majeur: rapidité,
précision, fiabilité...

On pourra citer le développement de codes de calculs parallèles, ou l'amélioration des techniques d'optimisation 
par les algorithmes génétiques.

\medskip
Nous avons parlé des modèles multi-échelle essentiellement pour les matériaux, mais le développement de modèles 
pour les structures minces (plaues, coques) rentre dans cette catégorie... et si les théories de plaques sont aujourd'hui
assez bien maîtrisées (d'un point de vue théorique et dans les codes), il reste des problèmes ouverts concernant
les coques minces.

\medskip
Les problèmes inverses constituent également un défi d'intérêt.
Ils permettent par exemple de remonter aux caractéristiques d'un matériau (qui seront utilisées dans d'autres
calculs), à partir d'essais sur un échantillon.

C'est également souvent la seule voie d'identification des paramètres et comportement pour tout ce qui
concerne la biomécanique et la modélisation du corps humain en général, car il n'est pas possible d'effectuer
des mesures réelles.

\medskip
Enfin, et peut-être surtout, les modèles développés aujourd'huis se veulent de plus en plus réalistes.
Il devient alors indispensable de coupler des modèles numériques différents, ce qui n'est possible
que si, au préalable, on a su établir des passerelles, des lieux de travail commun, entre des
spécialistes de plusieurs disciplines.

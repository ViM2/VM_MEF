\chapter{Le maillage}\label{Ch-mesh}
\begin{abstract}
À ce niveau du document, on peut considérer que la méthode des éléments finis a été présentée, au moins en ce qui concerne les aspects les plus classiques (et même un peu plus).
Nous avons décidé, avant d'entrer dans le détail de «subtilités» liées au comportement des matériaux et à la non-stationnarité, d'insérer ici un petit chapitre sur le maillage, dont les techniques de construction n'ont rien de commun avec celles relatives aux éléments eux-mêmes.
De plus, nous nous restreindrons aux maillages de type Voronoï\index[aut]{Voronoï (Gueorgui Feodossievitch), 1868-1908, Russe}-Delaunay.\index[aut]{Delaunay (Boris Nikolaïevitch), 1890-1980, Russe}
\end{abstract}

\medskip
\section{Introduction}

L'opération de maillage consiste à discrétiser un domaine (i.e. un milieu continu ou plutôt sa modélisation géométrique) par des éléments 
(éléments finis nous concernant), si possible bien proportionnés: 
au paragraphe \ref{Sec-mesh-carac}, nous avons déjà présenté les dimensions géométriques représentatives d'un maillage que sont 
le diamètre maximum des éléments $h$\index{diamètre d'un élément fini} et le facteur de forme du maillage\index{facteur de forme du maillage} 
$\sigma$, ainsi que le diamètre $h_K$ d'un élément $K$\index{diamètre d'un élément fini} et sa rondeur $\rho_K$\index{rondeur d'un élément fini}.
Ces deux derniers paramètres sont représentés sur la figure \ref{h-rho}. 

\medskip
Dans ce chapitre, nous traiterons du cas bidimensionnel.
Nous expliquerons les bases théoriques, mais ne rentrerons pas dans les détails pratiques de la programmation
d'algorithmes de maillage (il y a de très bons cours disponibles sur le sujet).

\medskip
Dans la suite nous aurons besoin des notions suivantes:
\begin{itemize}
   \item un \textcolorblue{segment fermé} (resp. ouvert) d'extrémités $a$ et $b$ de $\RR^d$ est noté $[a,b]$ (resp. $]a,b[$);
   \item un \textcolorblue{convexe} $E$ est un ensemble tel que: $\forall \intervalle{a}{b}\in E^2, [a,b]\subset E$;
   \item le \textcolorblue{convexifié} d'un ensemble $E$ de points de $\RR^d$, noté $\mathcal{C}(E)$ est le plus petit convexe
	contenent $E$.
   \item un domaine $\Omega$ (i.e. un ouvert de $\RR^d$) est dit \textcolorblue{polygonal} si son bord $\Gamma=\partial\Omega$
	est formé d'un nombre fini de segments;
   \item un \textcolorblue{n-simplex $(x_0, ..., x_n)$} est le convexifié des $n+1$ points de $\RR^d$ affine indépendant.
	Cela implique que $n\le d$. On a donc: un segment est un 1-simplex, un triangle un 2-simplex, un tétraèdre est un 3-simplex.
	Les sommets sont des 0-simplex.
\end{itemize}


\medskip
\section{Maillage simplexial}




\medskip
\section{Diagramme de Voronoï}



\medskip
\section{Maillage par projection}



\chapter{L'acoustique}\label{Ch-Acous}
\begin{abstract}
En complément avec ce qui a été vu aux chapitres précédents~\ref{Ch-temps} et~\ref{Ch-ondes} sur les problèmes non stationnaires et les ondes, nous allons nous focaliser dans ce chapitre sur l'acoustique, et plus particulièrement sur le calcul de problèmes pour lesquels les fréquences restent inférieures à quelques milliers de Hz. Nous en profiterons pour effectuer une présentation pratique de l'acoustique et des solutions qui peuvent être mises en œuvre physiquement.
\end{abstract}

\medskip
Dans le paragraphe~\ref{Sec-choc}, nous avons indiqué que la méthode des éléments finis devait se cantonner aux basses fréquences, celles-ci allant jusqu'à 600~Hz environ. Il est vrai qu'au-delà, de nombreuses autres méthodes existent... et nous souhaitions profiter de ce paragraphe pour les présenter.
Néanmoins, l'augmentation rapide des capacités des ordinateurs, fait qu'il est tout à fait raisonnable aujourd'hui de traiter des cas allant jusqu'à quelques milliers de Hz, disons 3000~Hz pour fixer les idées, à l'aide de la méthode des éléments finis. Cela permet de traiter la plupart des cas pratiques qui se posent à nous, et nous verrons dans ce chapitre pourquoi il est souvent inutile de monter plus haut en fréquence.

\medskip


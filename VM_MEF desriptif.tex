Le but de ce document est de présenter, si possible de manière simple et accessible, la théorie mathématique derrière la méthode des éléments finis afin que les ingénieurs utilisant cette méthode puissent en envisager toutes les applications.<br>
Dans une première partie, un survol des notions mathématiques est réalisé, puis le traitement du problème continu est discuté dans la partie II: cela constitue l'ossature théorique nécessaire à assoir la MEF sur un socle solide.<br>
La méthode des éléments finis à proprement parler n'est abordée qu'ensuite, à la partie III, complétée par des aspects plus "sophistiqués": homogénéisation, problèmes non stationnaires, ondes, non linéarités matérielles et géométriques, rupture. Cette partie se termine sur une brève présentation des méthodes connexes dont il faut avoir entendu parlé (BEM, Meshless, XFEM...), ainsi que sur une mise en garde concernant les singularités. <br>
v2: Les exemples (un peu plus nombreux) ont été insérés dans le cours du texte. Ils sont traités avec Ansys, Cast3M (castem) et FreeFem++. 
Désormais, deux versions existent de ce document: la "version cours" (plus colorée et dont les notations correspondent à ce que nous utiliserons en cours) et la "version livre" (plus classique et sage dans sa forme). Ces deux versions sont produites à partir du même code source.<br>
v3: Un chapitre sur le maillage et un sur les éléments finis stochastiques ont été ajoutés. Pour les étudiants, c'est cette dernière version qu'il faut prendre en compte car la stochastique fera partie des points abordés cette année.<br>
v4: Un chapitre sur l'acoustique a été ajouté.<br>
v5: Ajout de quelques compléments et exemples, ainsi que d'un chapitre sur l'optimisation<br>
v6: Quelques corrections et apports mineurs<br>
v7: Quelques corrections et apports mineurs (Ch 13 et 21 surtout)<br>
v8: Quelques corrections et apports mineurs (Ch 12 surtout)<br>
Nous remercions Mathias Legrand pour ses conseils avisés et sa relecture pertinente.